\newpage
\section{Konklusion}
Opgaven her beskriver i sin enekelthed udviklingen af spionage, fra Sun Tsu's tid til den moderne æra, med særligt fokus på Den Kolde Krig, og Stasi's omfattende overvågningsaktiviteter. Der er i den redegørende del blevet redegjort for hvilke typer af overvåning Stasi anvendte, samt hvordan Stasi både arbejdede indlands og udlands. Herunder har opgaven været indeomkring Guillaume's affæren. Samtidig adresseres de filosofiske perspektiver på privatlivets fred, som understreger behovet for en balance mellem national sikkerhed og individuelle rettigheder. John Locke's idéer om privatliv og ejendom spiller en central rolle i diskussionen om regeringens ansvar for at beskytte borgernes rettigheder. Endelig præsenteres et spyware-program som et praktisk eksempel, mens etiske og juridiske aspekter af spionage bliver diskuteret for at fremhæve vigtigheden af beskyttelse af privatlivets fred i et demokratisk samfund. Alt i alt anses det for at problemformuleringen er blevet besvaret, og at opgaven har opfyldt sit formål.
\section*{Resume}
Spionage har gennemgået en betydeligt udvikling siden Sun Tsu's tid, hvor strategisk brug af agenter var centralt. Under Den Kolde Krig  blev spionage en vigigt og central del i den ideologiske kamp mellem det komunistiske øst og det demokratiske og liberale vest. I de to dele ønskede man at vide hvad de andre havede gang i derfor blev efterretningstjenester som Stasi opprettet. Stasi var en af de nok mest effektive efterretningstjenester. Stasi infiltrerede højtstående politiske kredse, som demonstreret ved Guillaume affæren. Når der er tale om spionage kan man ikke undgå at tale om privatlivets fred, samt de af John Locke beskrevede naturlige rettigheder. Netop disse rettigheder må andses for at være grundlæggende menneskerettigheder i et liberalt samfund. Disse rettigheder bør derfor for alt verden beskyttes. Locke argumenterede for individets medfødte rettigheder til liv, frihed og ejendom, og betonede behovet for en begrænset regering, der respekterer disse rettigheder. Et eksempel på et spyware program viser, hvordan moderne teknologi kan bruges til overvågning. Programmet, skrevet i Python, kan tage skærmbilleder og webcam billeder, men har begrænsninger, der gør det ineffektivt til professionel spionage, men er stadig en god måde at fremvise hvad spywere kunne være. Brugen af programmer som disse må dog fremhæve en del etiske og moralske dilemmaer. Blandet andet om privatlivets fred. For at finde den af Arestoles beskrevede middelvej må man derfor gribe fat i konkrete beskrivelser. Og der skal vurderes fra sag til sag. Dog må det anses for at proportionalitets princippet er en god måde at vurdere om en handling er etisk korrekt. Dette princip er også en del af den danske lovgivning. Derfor må det anses for at være en god måde at vurdere om en handling er etisk korrekt.

\newpage
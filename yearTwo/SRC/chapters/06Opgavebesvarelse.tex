\newpage
\section{Opgavebesvarelse}\label{sec:OpgaveBesvarelse}

\subsection{Fra Hemmelige Øjne til de digitale Skygger}\label{sec:Redegorsel}
Når man høre om spionage, tænker man nok på James Bond, eller på anden måde hardcore typer der sniger sig rundt i mørket og opsamler oplsyninger, de er praktisk talt udødelige. Sådan er der dog ikke i virkligheden. Her er det ganske normale mennesker der, har fået et noget utraditionelt job. Man kan være spion på mange forskellige måder. Man kan være spion for et firma (Corporate espionage), en kriminel organisation, eller en stat. I denne opgave vil man dog fokusere på statlig spionage. \\
Man skal derfor tilbage til cirka 500 år f.v.t. Her skrev den kinesiske strateg og hærfører Sun Zi om vigtigheden af at bruge agenter og dobbeltagenter for at kunne vinde angreb, men også for at komme angrebene i forkøbet. Sun Zi sagde: "One who knows the enemy and knows himself will not be endangered in a hundred engagements."  % Kilde
Værket hedder "13 afsnit om militærstrategi" og er i dag en af de mest populære bøger om strategi. Værket lægger vægt på de politiske hensyn i krigsførelse og er understøttet af filosofiske og metafysiske overvejelser. % Kilde
Efter Sun Zi's tid har spionage ændret sig en del, og der er sket en markant forandring. Den tid der for opgaven her er mest interesseret, er dog tiden omkring den kolde krig. Under Den Kolde Krig var spionage ikke blot en sprøgsmål om militær efterretning, men en central del af den ideologiske og politiske kamp mellem Øst og Vest. Den østtyske sikkerhedstjenste, Stasi  anno 1949-1990 (Ministerium für Staatsicherheit) var en af de nok mest effektive og frygtende efterretningstjenester i verden. Stasi havede en bred og ommenfattende overvånings strategi. Hvor man både tillod at aflytte befolkningen men også udlandske personer såsom diplomater. Stasi's arbejde involverede ikke kun traditionel spionage som skygning, muldvarper, og sexspionage.% Kilde
De anvendte også avanceret udstyr til at overvåge. Under Stasi's tid begynder udruldningen af aflytningsudstyr for alvor. Man kan nu gemme mikrofoner i vægge, lamper, og meget andet. Deres arbejde var omfattende og man havde derfor en stor mængde af agenter. Man anslår at der var 1 Stasi agent for hver 166 østtysker. Dette er en meget stor mængde. Altså var der cirka 90.000 Stasi agenter i østtyskland. Mange af dem var enten friviligt eller under tvang. Stasi's indflydelse strakte sig langt ud over DDR's grænser. En af de mest bemærkelsesværdige operationer var infiltreringen af Vesttyskland. Et godt og konkret eksempel på netop dette var Stasi's arbejde under en operation der omhandlde den vesttyske agent Günter Guillaume. Operationen kendes også under Navnet Gulliamen affæren. Günter Guillaume arbejdede som rådgiver for Vesttysklands kansler Willy Brandt. Guillaume var en hemmelig Stasi-agent, der i mange år leverede følsomme oplysninger til Østtyskland. Hans afsløring i 1974 førte til en stor skandale og tvang Willy Brandt til at træde tilbage som kansler. Denne episode demonstrerer Stasi's evne til at infiltrere højtstående politiske kredse og påvirke den vesttyske regering. % Kilde

\subsubsection{Hvad med det der privatliv?} % Skriv konkret om John Locke og hans tanker om privatlivets fred
Privatlivets fred er en grundlæggende menneskerettighed, der omfatter individets ret til at leve sit liv uden unødvendig indblanding fra staten, andre borgere eller private organisationer. Det indebærer beskyttelse af personlige oplysninger, kommunikation og ejendom samt retten til at træffe personlige beslutninger uden at blive overvåget eller kontrolleret. Denne rettighed er afgørende for individets frihed og værdighed, og den er beskyttet af internationale menneskerettighedskonventioner, såsom FN's Verdenserklæring om Menneskerettigheder og Den Europæiske Menneskerettighedskonvention. John Locke, en af de mest indflydelsesrige filosoffer i oplysningstiden, bidrog væsenligt til at definere forståelsen af privatlivets fred. Locke definerede privatlivets fred i gennem sine værker. Locke argumentede for at mennesker i dens natur er klog nok til at kunne træffe beslutninger for sig selv, og at man derfor måtte lade mennesket værne om sig selv. Han definerede grundlæggende rettigheder så som retten til livet, frihed, og ejendom. Locke definerede desuden at dette ikke var rettigheder en fyrste eller et demokrati udstedet men at det var medfødte rettigheder som alle mennesker har. I sit værk "Two Treatises og Goverment" (1689) definerer Locke privatlivets fred som en del af retten til ejendom. For Locke omfatter ejendom ikke kun matrielle gendstande, men også ens person, handlinger og ideer. Han mente, at individets frihed og ejendom slulle beskyttes mod vilkårlig indblanding fra staten, eller andre individer. Ifølge Locke er formålet med regeringen at beskytte disse rettigheder, og hvis en regering ikke beskytter disse eller krænker dem, er det en hvermands ret at vælte regeringen. Lockes filosofi understrefer vigtigheden af en begrænset regering, der respekterer individets rettigheder, og handler inden for lovens rammer. Han mente, at love skulle baseres på samtykke fra de valgte/regede og skulle tjene til at beskytte deres naturlige rettigheder. Dette princip er grundlaget for moderne liberale demokratier og deres respekt for privatlivets fred. I den danske grundlov er retten til privatlivet beskyttet, og det er ulovligt for regeringen at indsamle oplysninger om borgerne uden samtykke eller en dommerkendelse. Desuden er retten til ejendom også beskyttet, dette er nedfæsteret som den private ejendomsret. \emph{"§ 73. Stk. 1. Ejendomsretten er ukrænkelig. Ingen kan tilpligtes at afstå sin ejendom, uden hvor almenvellet kræver det. Det kan kun ske ifølge lov og mod fuldstændig erstatning."}
Privatlivets fred, som Locke definerede det, har direkte relevans for spørgsmålet om statslig spionage og overvågning. Efterretningstjenesters overvågningsaktiviteter kan udgøre en trussel mod individets rettigheder, hvis de ikke udføres inden for rammerne af loven og med passende tilsyn. Balancen mellem national sikkerhed og beskyttelsen af borgernes privatliv er en kompleks udfordring, der kræver omhyggelig afvejning af behovet for sikkerhed og respekten for individuelle frihedsrettigheder. Lockes tanker om privatlivets fred understreger, at selvom regeringen har en legitim interesse i at beskytte nationens sikkerhed, må dette ikke ske på bekostning af individets grundlæggende rettigheder. Enhver form for overvågning skal derfor være nødvendigheds- og proportionalitetsafprøvet, og der skal være mekanismer på plads for at sikre, at borgernes rettigheder respekteres og beskyttes.

\subsection{Spywere program}
Der er blevet programeret et spywere program med forskellige funktioner. Programmet er skrevet og udviklet i sproget python, man har anvendt funktionel programmering til at skrive programmet. Man har valgt at anvende funktionel programmering da dette vil være den smarteste løsning. Hvis man brugte fx objektorientering vil man støde ind i det problem at når programmet bliver udløst så vil det give et drop i hvor hurtigt computeren er. 80 årige Bente vil måske ikke opdage det, men professionelle spioner og data analytikere, vil lyn hurtigt opdage at der hver 10 sekund bliver oprettet en ny instans af programmet. Dette vil dog ikke ske i samme grad når man anvender funktionel programmering. Her vil instansen nemlig stadig være aktiv. Og man vil ikke opdage at programmet direkte bliver udløst mens man anvender programmet. Programmet er cirka lige så godt som cæsar kryptering er på computerer nu tildags. Med andre ord det er slet ikke godt, og man kan ikke bruge det til noget efterretningsarbejde. Programmet er dog stadig et godt eksempel på hvordan og hvad form for data der indsamles. 
\newpage
\subsection{Diskussion og Vuderering}



\newpage
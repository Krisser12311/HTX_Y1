\newpage
\section{Opgavebesvarelse}\label{sec:OpgaveBesvarelse}

\subsection{Fra Hemmelige Øjne til Digitale Skygger}\label{sec:Redegorsel}
Når man høre om spionage, tænker man nok på James Bond, eller på anden måde hardcore typer der sniger sig rundt i mørket og opsamler oplsyninger, de er praktisk talt udødelige. Sådan er der dog ikke i virkligheden. Her er det ganske normale mennesker der, har fået et noget utraditionelt job. Man kan være spion på mange forskellige måder. Man kan være spion for et firma (Corporate espionage), en kriminel organisation, eller en stat. I denne opgave vil vi dog fokusere på statlig spionage. Man skal derfor tilbage til cirka 500 år f.v.t. her skrev den kinesiske strateg og hærføre Sun Zi om vigigtigheden af at bruge agenter og dobbeltagener for at kunne vinde angreb. Men også for at komme angrebene på forhånd. Værket hed "13 afsnit om militærstrategi" Og er i dag en af de mest populærer bøger om strategi, værket lægger vægt på de politiske hensyn i krigsførsel og er understøttet af filosofiske og metafysiske overvejelser. 


\newpage
\subsection{Diskussion og Vuderering}



\newpage
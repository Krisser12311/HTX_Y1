\section{Eksammens opgave 1 - Lys}\label{sec:eks1}
\subsection{a Beregn lysets frekvens i luft når bølgelængden er 560nm}
For at finde frem til frekvensen af lys ved 560nm, skal man bruge formlen for hastigheden først det skal man da man ikke ved hvad svingningstiden er.
\begin{equation}
    V=\lambda/T
\end{equation}
\begin{equation}
    V=\lamda * f
\end{equation}
\begin{equation}
    f=\frac{V}{\lambda}
\end{equation}

\begin{equation}
    f=\frac{3*10^8 m}{560*10^-9 m}
\end{equation}
\begin{equation}
    f=5.357*10^14 Hz
\end{equation}
\subsection{Beregn brydningsvinklen når indfaldsvinklen er 15. Lav en skite der ilustrerer brydningen}

\subsection{Find den kritiske vinkel hvor der totalreflektionen indtræffer}
Et nyt forsøg laves hvor lysstrålen sendes fra sprit op i luften. 
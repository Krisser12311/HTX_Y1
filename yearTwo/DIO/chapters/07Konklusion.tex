\newpage
\section{Konklusion}
Denne opgavebesvarelse er blevet udført ved hjælp af en struktureret tilgang, der først of fremmest begyndte med en redegørelse for udviklingen af atombomben og dens anvendelse under Anden Verdenskrig. Derefter blev der lavet en argumentationsanalyse af en tale fra den amerikanske præsident Hary Truman, der redegjorde for brugen af atombomben. Denne analyses primærer fokus var at undersøge de virkemidler Truman brugte altså hvilke retoriske virkemidlder han bruge, hvordan han anvendte sroget til at levere en påstand med både belæg og med styrkemarkører. Endeligt blev der foretaget en diskussion og vudering af de etiske dilemmaer omkring brugen af atombomber under Anden Verdenskrig. Dette blev udført ved hjælp af forskellige filosofiske perspektiver, herunder Aristoteles dydsetik og Immanuel Kants pligtetik. Disse perspektiver blev brugt til at udforske om det var etisk forsvarligt at bruge atombomberne. Samlet set kom opgaven hele vejen rundt om problemformulering og har derfor givet et godt indblik i atombombens historie og dens etiske dilemmaer.

\newpage
\section{Opgavebesvarelse}\label{sec:OpgaveBesvarelse}

\subsection{Redegørelse}\label{sec:Redegorsel}
Når man tænker på atomvåben, kommer man ikke uden om Den Kolde Krig, Anden Verdenskrigs afslutning og frygten for en reel atomkrig og menneskehedens udryddelse. Verden står stadig over for truslen om brugen af dette ødelæggende våben. Men hvad er historien bag dette våben? Hvordan blev det til, og hvordan er det blevet brugt tidligere? \\
Tilbage i 1938 opdagede tyske fysikere Otto Hahn, Fritz Strassmann og Lise Meitner, at bombardering af uran med neutroner producerede et bariumisotop som et spaltningsprodukt. (\cite{MaxPlanckInstitute}) I august 1939 sendte Einstein et brev til præsident Roosevelt og advarede om Tysklands undersøgelser af uran i store mængder, hvilket kunne føre til våbenproduktion. Han anbefalede tæt kommunikation mellem regeringen og forskere i USA.
Kort efter Einsteins brev (\cite{EinsteinLetter}) invaderede Tyskland Polen, og Anden Verdenskrig begyndte. USA blev involveret efter Japans angreb på Pearl Harbor. En rapport fra "The Advisory Committee on Uranium" konkluderede, at atombomben kunne ændre krigens udfald. (\cite{timeline}) Altså kunne A-våben lige pludslig have indfyldese på Velstand, velfærd og genrelt magt det bar selvfølig stadig et vis ansvar at have adgang til sådan et våben. Denne repport førte til oprettelsen af Manhattan-projektet, der involverede tusindvis af forskere. Manhattan-projektet blev delt mellem tre centre:  Los Alamos, Oak Ridge og Hanford. (\cite{NPS}) Man delte det op i tre dele fordi man havde to veje man gerne ville gå. Man ville gerne prøve at udvikle en Plutonium bombe og en Uran bombe. Los Alamos som var \emph{"center of operation"}. Los Almos var ledet af Robert Oppenheimer. Ved Los Almos skulle de primæret fokuserer på forskning og forsøgsarbejde. Oak Ridge skulle producerede Uran-235 isotopen, og Hanford skulle producerede Plutonium. Atombomben er blevet anvendt to gange i krigssituationer og ellers er der lavet nogle testspringer på dem. De to gange hvor atombomben er blevet andvent var under Bombningen af Hiroshima og Nagasaki. Det var noget, der blev sat i værk efter et forsøg på en operation kaldet "Operation Downfall". (\cite{OperationsDownfall}) Operation Downfall var en plan om at invadere Japan. Japan var på daværende tidspunkt en del af aksemagterne og var den sidste aksemagt, som ikke havde overgivet sig. Det var på tale at invadere hele Japan. Men efter intense kampe på øerne Okinawa og især Iwo Jima, både fra militær front men også fra dele af befolkningen, estimerede man at invasionen af hovede øen Japan ville komme til at koste mellem 1,7 og 4 millioner ofere/sårede. Samt 5 til 10 millioner døde Japanere. Truman ville have at Japan overgav sig og derofr valgte at udsende en advarsel til Japan kaldt "The postdam declaration". I denne advarsel stod der at hvis Japan ikke overgav sig, så ville de blive udsat for et ny og meget ødelæggende våben. Japan valgte dog at ignorere advarslen. Elleve dage efter advarslen forlatte flyet "Enola gay" basen Tinian, og fløj mod Hiroshima. Den 6 august 1945 kl 8:15 faldte atombomben Littel boy over Hiroshima cirka 600 meter over byen eksploderede den og endte med at koste 80.000 mennesker livet. Flere dage efter så man, at flere døde af radioaktivitet. Japan valgte dog stadig at de ikke ville overgive sig, og den 9 august 1945 faldte der en ny atombombe over Nagasaki. Denne gang var det atombomb Fat man som blev brugt. Denne gang døde der 40.000 mennesker. Japan overgav sig dog stadig ikke, og det var først da Sovjetunionen erklærede krig mod Japan, at Japan overgav sig.

\newpage
\subsection{Analyse}
Følgende er en analyse af Trumans tale den 6. august 1945, (\cite{TrumanSpeech})
hvor han fortalte om bombningen af Hiroshima. Trumans tale griber fat i de politiske, militære og moralske overvejelser. Der lå til grund for beslutningen om at bruge et altødelæggende våben. Brugen af atomvåben rejser store spørgsmål og kræver dybe overvejelser i forhold til etik. Mange af disse etiske dilemmaer er stadig genstand for debat i den verden, vi lever i nu. Et dilemma, som kunne være genstand for debat, kunne være: Hvornår kan det retfærdiggøres at bruge et sådant altødelæggende våben? Samt hvilke konsekvenser har det for de mennesker, som bliver ramt af det? Trumans tale omhandlede lige præcis dette. Hvordan man havde overvejet situationen og valgt at bruge atombomben. Du vil stadig kunne diskutere om denne beslutningen var den rigtige \\
\emph{(...) The battle of the laboratories held fateful risks for us as well as the battles of the air, land and sea, and we have now won the battle of the laboratories as we have won the other battles (...)} 
Talen antyder, at kampen i laboratorierne udgjorde betydelige risici, men at disse risici blev overvundet på samme måde som andre kampe. Truman har i bund og grund ret i dele af denne påstand. Kernefissionens ukendte konsekvenser skabte usikkerhed om, hvorvidt den ville føre til en enorm eksplosion eller endda udslette jorden. Belægget for denne påstand ligger i sammenligningen mellem laboratorieeksperimenter og kampene på slagmarken og i luften, hvilket fremhæver laboratoriernes indsats som næsten lige så vigtige som de mere synlige slag. Belæget lyder som følgende: \emph{"The battle of the laboratories held fateful risks for us as well as the battles of the air, land and sea"} Dette er belæget da det direkte understreger at kampen i laboratorierne var lige så ricikabel som kampene i luft, land og havneområderne. Ved at sammenligne disse forskellige former for kamp understreger det citatets hovedpointe og viser, at laboratorieindsatsen var af afgørende betydning for krigen. Da talen kommer fra den daværende amerikanske præsident, bærer den en hvis autoritet, hvilket gør det klart, at Truman's ord skal tages alvorligt, og at alle forstår alvoren i situationen. Truman argumenterer også for, at uden udviklingen, og brugen af atombomben ville fred aldrig være opnået. Hvilket delvist berettiger dens anvendelse. På denne måde markerer Truman den store risiko og indsats, der blev lagt i laboratorierne. En potentiel modsigelse kunne være, at sejren i laboratorierne ikke direkte kan sammenlignes med sejrene på slagmarken, da de repræsenterede forskellige former for indsats og risici. Der kan også være etiske bekymringer ved udviklingen af atomvåben, hvilket ikke nødvendigvis gør sejren til en ubetinget "vundet kamp". styrkemarkøren i dette citat må være "fateful"  direkte oversat til dansk må det oversættes til skæbnesvangre. At noget er skæbnesvangre fortæller ganske kort at det hele er usikker og at det kan både gå godt men også skidt. Det er derfor med til at sikre at alle forstår hvor farlig og vigtigt udviklingen af atombomben har været. Samlet set følger citatets argumentation Toulmins model ved at præsentere en påstand, støtte den med belæg og hjemmel, understrege dens relevans med rygdækning og styrkemarkører og adressere potentielle modsigelser gennem gendrivelse. \\
\emph{(...) Sixteen hours ago an American airplane dropped one bomb on Hiroshima, an important Japanese Army base. That bomb had more power than 20,000 tons of T.N.T. It had more than two thousand times the blast power of the British "Grand Slam" which is the largest bomb ever yet used in the history of warfare. (...)} \\
Dette citat fremhæver noget helt andet end det tidligere citat. Dette citat fremhæver at atombomben var en meget kraftig bombe, og at den var meget kraftigere end nogen anden bombe der nogensinde var blevet brugt. Belæget for denne påstand er at Truman direkte siger at bomben var stærkere end nogen anden bombe der nogensinde var blevet brugt, det kan ses ved dette citat \emph{"It had more than two thousand times the blast power of the British "Grand Slam" which is the largest bomb ever yet used in the history of warfare."} Ligesom med det anden citat må det anses for at være troværdigt, da det er fra en tale af den daværende præsident. styrkemarkøren i dette tilfælde er \emph{hat bomb had more power than 20,000 tons of T.N.T.} det er det fordi det er med til at sætte tal på hvor farlig og hvor dødbringde denne bombe var. Gendrivelsen vil ligge i at modstandere af påstanden kunne hævde, at der ikke var behov for at bruge sådan en kraftig bombe mod Japan eller at måden, hvorpå krigen blev ført, var umoralsk. Dette citat fokuserer primært på de tekniske fakta om bomben og dens kraft, uden at adressere de politiske eller etiske spørgsmål.
Griber man fat i talene genrelt vil man også kunne se at Truman tænker over brugen af retoriske virkemidler, han formår at bruge både Ethos, Pathos, og Logos. Altså taler Truman til vores Mave, Hjerne og Hjerte. Han har automatisk Ethos da han er præsidenten for USA, og derfor har en vis autoritet. Han bruger Pathos ved at han appelere til følelserne hos lytterne  især ved at skabe en følelse af frygt og alvor omkring konsekvenerne af atombombes udviklingen og anvendelse. Han bruger ord som "fataful risks" (skæbnesvangre risici) til at understrege alvoren i situation og skabe en følelse af uro og bekymring hos lytterne. Desuden taler han om "risks" (risici), hvilket kan skabe angst og usikkerhed omkring fremtiden. Han bruger Logos ved at han taler om de tekniske fakta om bomben. Den måde Truman "leger"  med sproget gør at han skaber en følelsesmæssig forbindelse med lytterne og forstæker  forstærker budskabets alvorlighed. Dette kan dermed overbevise lytterne til at forstå at han altså har taget en chance og den chance er han lykkedes med. Logos findes i det at han taler om atombombens styrke og på anden måde omtaler fakta. 

\newpage
\subsection{Diskussion og Vuderering}
I kølvandet på de altødelæggende bombardementer af hhv. Hiroshima og Nagasaki under Anden Verdenskrig er etikken omkring brugen af atomvåben, og den form for krigsførsel blev diskuteret, men hvordan kan man vurder om en våben er okay a bruge i krig? Hvordan kan filosofiske perspektiver hjælpe os med at forstå netop det? Og hvordan kan man vurdere om det var okay at bruge atombomben? Et af de centrale spørgsmål når det kommer til at bruge et våben i krig er om man kan retfærdiggøre brugen af det. Til at kunne besvare netop dette skal man gribe fat i hvordan filosoffer definerer moral og etik, til det kunne man bruge Emanuel Kant, Aristoteles og Jeremy bentham. Til en start kan man gribe fat i Aristoteles og hans dydsetik, som handler om at finde den gylende midelvej. \\
Hvordan passer brugen af et ekstremt altødelæggende våben ind i dette? Ville Aristoteles anerkende brugen af atombomben? Det er svært at vide! Men man kan prøve at sætte det i perspektiv. Hvis man skal sætte det i perspektiv, ville man opstille det gode mod det onde. Det var ondt at dræbe så mange mennesker, som nok ikke hvade noget med krigen at gøre. Det gode kan opstilles som om at der ikke var flere der døde. Skal man tro på de tal amrikerne selv har opstillet, ville mindst 5 milioner døde Japanere og mindst 1,7 milioner ofere. Det endste onde ved brugen af A-våben var ikke kun at mange døde men også at mange døde på en meget grusom måde. Tager man det i betragtning lyder det nok som om at det var godt hvis man kun tæker på dydsetikken. Det var med til at sætte en stopper for krigen. Verden er ikke sort-hvid og man skal derfor tage flere perspektiver med. På samme måde som man har anvendt dydsetikken kan man også andvende Kant's etik. Kant lagede vægt på pligt og moral. Om Kant ville argumenterer for brugen af atombomben var i overensstemmelse med pligtetikken og menneskelig værdihed, eller om han ville se det som en overtrældese af grundlæggende principper for moral? Det er svært at vide, men man kan prøve at opstille det og se på de forskellige sider af sagen. Kant ville sandsynligvis argumentere for at brugen af atombomberne ville være en overtrædelse af de grundlæggende principper for moral og menneskelig værdighed. Kant argumenterede for, at mennesker altid skal behandles som de selv ønsker at blive behandlet.  Mennesker er ikke blot et midl som bruges til at opnå bestemte mål, de er individer som har føleser og som selv har en holding. Kant mente at man ikke skulle gøre noget baseret på føelser og drifter. Det skydes at  kant mente altså at man skulle tage beslutninger baseret på viden og fakta. Det skydes at Kant mente at man tager dumme beslutninger hvis man kun bassere ens beslutningsgrundlag på føleser.Derfor ville kant nok sige at brugen af atombomber, ville medførte mange uskyldige døde og at de ville lide indtil at de ville død. Men at det faktuelt ville være godt. Hvis altså at man skal tro på de amerikanske tal og hvis man tror på det imparativ om at krigen aldrig ville have endt ellers. Kant vil derfor nok betragte bombningerne af Hiroshima som en klar beslutning, det var okay! Til sidst kan man kigge på Nytteetikken, som blev opstillet af Jeremy Bentham. Nytteetikken beskriver hvordan det kan give lykke til flest mulige mennesker. Et sted man finder nytteetikken kunne være hvad vedrører medecin, at alle dansker for en gartis Corona vaccine, gør at mange dansker kan 1. leve længere 2. ungå at blive syge, dette lever derfor op til nytteetikken. Hvis man kigger på en dverg som kan blive 10 cm højre af at få medecin til 2 milioner om året vil dette ikke blive dækket af nytteetikken da det ikke giver lykke til de mange men lykke til den enkelte. Men hvordan kan man bruge nytteetikken på bombningerne af Hiroshima og Nagasaki? Sætter man det op ligesom ved de andre senarier, så kan man argumentere for at bombningerne bidrog til at afslutte krigen hurtiger og dermed minskede det tab af mennske liv. Hvis man ikke hvade bombet de to byer kunne krigen have varet i lang tid og have kostet mange mennske liv. På den anden side kan man argumentere for. de langsigtede konsekvenser af atombomberne, herunder enorme tab af cevile mennesker som intet havde med krigen at gøre og de alvorlige milijømæssige konsekvneser, dette må siges at være meget vigtiger end at afslutte krigen hurtigere. Man kan hævde at atrombomberne ikke resulterede i den største samlede lykke eller nytte, da de forårsagede så meget lidelse og ødelæggelse. Bentham opfordrer os til at veje fordele og ulemper ved handlinger ud fra deres samlede konsekvenser for at bestemme deres moralske værdi. Når vi anvender dette perspektiv på brugen af atombomberne, bliver man nødt til at afveje de umiddelbare resultater af at afslutte krigen mod de langsigtede lidelser og konsekvenser, som brugen af sådanne våben medførte. Dette åbner op for en diskussion om, hvorvidt de umiddelbare fordele ved at afslutte krigen hurtigere opvejer de langsigtede konsekvenser af atombomberne. \\


\newpage
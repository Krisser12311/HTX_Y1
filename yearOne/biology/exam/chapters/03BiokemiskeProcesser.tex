\newpage
\part{Biokemiske processer}
    \section{Forklar de 2 biokemiske processer respiration og fotosyntese.}
        \subsection{Respiration}\label{sec:respiration}
            Hvad er Respiration? 
        \subsection{Fotosyntese}
            I en eukaryot plantecelle har man et organell som man ikke har i en eukaryot dyre celle, nemlig grønkorn. I grønkornene sker der en masse biokemiske processer som vi tilsammen kalder for fotosyntese. Under fotosyntese dannes to vigigtige stoffer, nemlig glucose og ilt. glucose (\begin{math}C_6H_{12}O_6\end{math}) er en sukkerart som er vigtig for planten, da den bruger den til at danne andre stoffer som den har brug for. 
            \begin{math}O_2\end{math} er ilt, som er vigtigt for alle levende organismer, da det er det vi bruger til at danne energi her i blandt \ref{sec:respiration}.
            I fotosyntese skal planten bruge to ting \begin{math}CO_2\end{math} og \begin{math}H_2O\end{math}. Alt det førnævte er ting som planten optager fra det miljø den er i. De to stoffer bruges til at skabe det organiske stof glucose. Glucose er et meget energirigt stof derfor kræve fotosyntese lysenergi. Heraf foto-syntese (lys drevet syntese)  \begin{math}H_2O\end{math} er vand, som planten optager fra jorden.
            Den simple formel for fotosyntese er: \begin{equation} 6CO_2 + 6H_2O \rightarrow C_6H_{12}O_6 + 6O_2 \end{equation}\label{eq:fotosyntese}
            Fotosyntese består af de 15-20 delprocesser, som overall kan beskrives med den fromel som ses ovenfor. (Se afsnit \ref{eq:fotosyntese})
            Man kan dele disse processer op i to dele hhv mørke- og lysprocessen. \newline 
            \subsubsection{Lysprocessen}
                Lys processerne kræver lys og kan kun foregå når der er lys til stede. Lysprocessen sker i thylakoiderne. I processen bliver der brugt \begin{math}H_2O\end{math} O som bliver lavet om til ilt og hydrogen. Hydrogen bliver hoverført til \begin{math}NADP^+\end{math} så det danner \begin{math}NADPH\end{math} (\begin{math}NADPH\end{math} er et organiske molekyle.) 
                Lysenergien anvendes til at sammensætte \begin{math}ADP + P_i\end{math} Og på den måde omdannes \begin{math}ADP\end{math} og \begin{math}P_i\end{math} til \begin{math}ATP\end{math} Denne process er afhængig af chloroflyl (Det som absorbere lysenergien) under lysprocessen dannes altså \begin{math}ATP\end{math} og \begin{math}NADPH\end{math} og ilt. En del af ilten bruges til respiration (se afsnit \ref{sec:respiration}), mens resten af ilten frigives til atmosfæren.

            \subsubsection{Mørkeprocessen}
                Mørkeprocessen er den del af fotosyntesen som ikke kræver lys. Denne proces foregår i grønkornene. Denne proces er en meget kompleks proces, som vi ikke vil gå i dybden med. Det vi skal vide er at den bruger energi fra lysprocessen til at danne glukose.


    \section{Redegør for øvelsen: Fotosyntese og Respiration hos vandpest.}
    \section{Diskuter hvordan de 2 processer kan have indflydelse på klimaet.}
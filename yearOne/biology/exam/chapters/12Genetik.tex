\newpage
\part{Genetik}
\subsection*{Forklar hvad der forstås ved nedarvning, herunder 1-gen dominant/recessiv nedarvning.}
Nedarvning er det område inden for genetik, der handler om, hvordan egenskaber går i arv fra forældre til afkom. Når man arbejder med nedarvning er det ligesom at arbejde med klassisk genetik. Klassisk genetik beskriver, hvordan et enkelt eller nogle få gener nedarves og kommer til udtryk hos organismer. Alle dipolide celler (Diploide celler, er celler som indeholder to kopier af hvert kromosom. Menneskeceller er diploide. Det modsatte af en diploid celle er en haploid celle.) i kroppen indeholder to versioner af samme gen, kaldet alleler. Det blev også gjort klart, at disse gener bliver udtrykt på forskellige niveauer. Der er nogle alleler, som er dominerende, mens andre er recessive. En dominerende allel vil typisk altid have prioritet over en recessiv allel og vil derfor komme til udtryk i fænotypen. Fænotype er % TODO

Genetik er dog ikke altid så simpelt, at man kan betegne alle alleler som enten recessive eller dominante. I tilfælde af øjenfarve er der faktisk 16 forskellige gener, der har indflydelse på hvilken fænotype, der bliver udtrykt og øjenfarve er derfor polygent, da der er flere gener som har indflydelse på fænotypen

\begin{longtable}{| m{3cm} | m{3cm} | m{3cm} |}
    \hline
    Skema & Brun & Blå \\ \hline
    Blå   & Bb   & bb   \\ \hline
    Brun  & BB   & Bb  \\ \hline
\end{longtable}

\begin{longtable}{| m{3cm} | m{3cm} | m{3cm} |}
    \hline
    Skema & X & Y \\ \hline
    X   & XY   & XY  \\ \hline
    X  & XX   & XY  \\ \hline
\end{longtable}
Denne tabel viser krydsningsskema om kromosomer hvor om ens barn bliver pige eller dreng. Drenge kromosomer har et par som hedder XY hvor kvinder har XX og i tabellen kan vi se at det er en 50/50 om det bliver en dreng eller pige.


\subsection*{Redegør for mutationer, samt hvordan de kan lede til variation.}
Mutationer er ændring af cellers DNA. Mutationer sker gennem hele cellens levetid og kan have mange konsekvenser. 

Dna er et makromolekyle med mange informationer, derfor når det skal kopieres og deles ved celledelingerne, kan der opstå fejl på grund af radioaktiv stråling, solskoldninger, eller kemiske stoffer, der ødelægger de kemiske bindinger i molekylerne. Når informationerne skades, vil resultatet ofte blive, at de proteiner DNA'et koder for, mister dens funktion. 

\subsection*{Diskuter hvordan fænotyper kan have betydning for evnen til at overleve. Tag udgangspunkt i øvelse om selektion}
Fænotype kan have en betydning for evnen til at overleve pga. fænotype er det der styrer de observerbare egenskaber eller karakteristika hos en organisme, der er bestemt af både deres genetiske arv og miljømæssige faktorer. Vi kan tage udgangspunkt i et forsøg som vi har lavet der omhandlede selektion, hvor vi havde hvide, sorte og røde perler. Derefter skulle vi vælge den perle vi først lagde mærke til efter at have lukket øjnene. 
\newpage
\part{Hormonregulering hos kvinder}
\subsection*{Forklar opbygningen og funktionen af de kvindelige kønsorganer}
\subsection*{Redegør for hvad hormoner er samt hvordan de transporteres og reguleres i kroppen.}
\subsection*{I forhold til øvelsen om prævention og sexuelt overførte sygdomme ønskes en redegørelse for præventions virkemåde. Kom desuden ind på fejlkilder i forsøget}
Både mænd og kvinder har mange forskellige kønsorganer med forskellige funktioner. Hos kvinder er der 13 forskellige kønsorganer, som har forskellige formål i kroppen. Klitoris, hymen, store kønslæber, små kønslæber, æggeledertragt, æggeleder, æggestok, livmor, livmorhals, blære, kønsben, urinrør og skeden. Klitorisen er ligesom penishovedet på den måde at den er tæt pakket med nerveceller, og kan derfor reagerer på lette berøringer. Skeden er klædt med en slimehinde og selve væggen er elastisk, når en kvinde indgår i samleje bliver der udskillet en væske som fungerer som et smøremiddel. Slimhinden producerer et kulhydrat som fremmer væksten af nogle mælkesyrerbakterier. De bliver brugt til at sørge for at uønskede bakterier ikke kan overleve i det sure miljø der er i skeden. Det primære kønsorgan for kvinder er æggestokkene, da det er der hvor æggene er anlagt. Ægcellerne liger i små væskefyldte blære som kaldes follikler. Hver måned bliver der modnet mange af æggene, dog er der typisk kun et æg der udvikles langt nok til at blive frigjort. Ægget samles op af æggeledertragten og føres hen mod æggelederen. I æggelederen vil ægget blive befrugtet og fortsætte mod livmoren, hvor det vil sætte sig fast i livmorvæggen. 
Cyklussen starter under puberteten og vil forsætte i 35-40 år efter. Som sagt sker der en modning af ægcellerne hver måned i folliklerne. Det sker da FSH for ægcellen til at vokse, og FSH stimulerer også follikelcellerne til at producere østrogen. Efter 14 dage vil der som regel kun være en follikel der er færdigudviklet og klar til ægløsning. Ægløsningen sker 14 dage efter der går hul på folliklen og det modne æg bliver frigivet. Ægget bliver frigjort når folliklen bliver stimuleret af LH, og på samme tid bliver follikelcellerne omdannet til det gule legeme hvor progesteron bliver dannet. Dog hvis ægget ikke bliver befrugtet så vil det gule legeme holde op med at fungere og menstruationen udløses.

Hormoner er stoffer der er nødvendige for en organisme, da de lader celler kommunikere med hinanden. Hormoner er som sagt stoffer, som bliver udskillet af specielle celler, og bliver brugt til at kommunikere med andre celler, der kan genkende hormonet. Celler kan genkende hormoner ved brug af hormon-receptorer. Hormoner bliver typisk lavet af endokrine kirtler, og hormoner bliver udskillet til blodbanen. Hos både mænd og kvinder bliver kønshormonerne primært udskillet fra kønskirtlerne, og selve reguleringen af hormonerne foregår mellem hypothalamus, hypofysen og kønskirtlerne. 
Det primære hormon hos kvinder er østrogen, dog er det ikke det eneste hormon kvinder danner, og bliver dannet på cirka samme måde som testosteron bliver dannet. Processen starter i hypothalamus hvor der bliver udskillet GnRH, som stimulerer hypofysen så der bliver dannet LH og FSH. FSH bliver brugt til at stimulerer æggestokkene, specifikt det gule legeme, så de kan danne østrogen og LH bliver også brugt til at stimulerer æggestokkene, dog så de kan danne progesteron. Ligesom hos mænd bliver koncentrationen af de primære hormoner reguleret ved brug af et negativt feedback loop. Når der er en høj koncentration af progesteron og østrogen i blodet vil hypofysens evne til at danne FSH og LH formindske. Faldet af koncentration af LH gør så det gule legemer henfalder, som fører til et fald i koncentrationen af østrogen og progesteron. Det betyder så også at hele processen kan ske igen da hypofysen kan igen begynde at danne LH og FSH til at stimulere æggestokkene og danne østrogen og progesteron igen.

I øvelsen om prævention og sexuelt overførte sygdomme skulle vi se hvordan kønssygdomme spreder sig og om præventionsmiddler hjælper med at stoppe spredningen af kønssygdomme. Vi brugt reagensglas som en erstatning af penissen. Kondom som vores præventionsmiddel, petriskåle med agar for at vi kunne tydeligt se hvor bakterierne spredte sig. Vi brugte også E.Coli som en erstatning af klamydia da det ikke ville have været sikkert for os at lege med en kønssygdom. Min gruppe kom dog til at lave fejl under forsøget da vores resultater gik helt imod alle andre gruppers. Den første fejl vi lavede var at vi pressede reagensglasset alt for hårdt mod agaren i petriskålene. Det gjorde så det var svært for os at aflæse resultaterne og har måske også revet hul i kondomen, hvilket er et selvfølge at det er en fejl, da forsøget handlede om præventionsmidler. Vi glemte også at tage stilling for at det kun var E.Colien der kom ind i petriskålene og vi regner med at der også har været noget kontaminering fra kimnedfald som har forvrænget vores resultater. 
Vi ved dog at en kondom burde fungere ved at sikre sig at der er ingen kontakt mellem det mandlige kønsorgan og andre steder. Det burde stoppe spredningen af kønssygdomme da det fungerer som en barriere og stoppe en overførelse af kropsvæsker mellem partnere. Dog er det jo ikke det eneste kondommer bliver brugt til da det også virker som en måde at forhindre gravidtitet, ved at, som sagt, forhindre at sæd bliver overført fra penissen til skeden.


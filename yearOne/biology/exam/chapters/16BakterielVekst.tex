\newpage
\part{Bakteriel vækst}
\subsection*{Redegør for bakteriecellers opbygning}
\subsection*{Forklar den bakterielle vækstkurve og kom ind på typen af celledeling, samt hvad der sker i de enkelte faser. Diskuter formål og forskelle mellem mitose og meiose}
\subsection*{I forhold til forsøget om osmose skal det diskuteres hvordan salt mm, kan anvendes til konservering, samt hvilken effekt det kan have på bakterier såvel som større organismer.}
Bakterierceller er encellede prokaryote organismer som har en cellemembran og cellevæg omkring. Bakteriecellen er opbygget ligesom mange andre celler, en cellemembran lavet af phosplipider, forskellige typer proteiner inde i cellen, cytoplasma med en masse forskellige organeller som bakteriekromosomet, plasmider og ribosomer. 
Plasmiderne er meget små cirkulære stykker DNA-molekyler som kan kopiere sig selv. Der er cirka 40-60 af dem per bakteriecelle. Bakteriekromosomot er også i stand til at kopiere sig selv ligesom plasmiderne, de er omgivet af proteiner og flyder rundt frit i cytoplasmaet. Ved ribosomerne foregår proteinsyntese af enzymer til cellen, stofskifte eller proteiner. 
Bakterier bliver typisk opdelt i to grupper, Gramnegative og Grampositive. Grampositive bakterier har en cellevæg lavet af et tykt lag peptidoglycan, som er opbygget af N-acetylglucosamin og N-acetylmuraminsyre, hvor alt holdes sammen af oligopeptider. Cellevæggen bliver bundet sammen med cellemembranen af lipoteichoinsyre. Gramnegative bakterier har også en cellevæg, dog består den af et tyndt lag peptidoglycan og en ydre membran. Den ydre membran bliver holdt sammen med cellevæggen med lipopolysaccharider og lipoproteiner.
Mængden af bakterier vil ikke kune øge forevigt, men vil i stedet begynder at falde og ofte ende med at mange af bakterierne vil dø. Der er typisk fire faser i bakterievæksten, nølefasen, eksponentiel fasen, stationær fasen og dødsfasen. Bakterier går ikke igennem mitose eller meiose ligesom andre celler da det jo kun er eukaryote celler der går gennem de to typer celldelinger. Bakterier går gennem noget som kaldes binær fission hvor cellen bliver spaltet og laver datterceller. Formålet med denne type deling er bare at formere.
\newpage
\begin{longtable}{| m{3cm} | m{14cm} |}
    \hline
    Fase & Beskrivelse \\ \hline
    
    Nøglefase & I nølefasen stiger antallet af celler ikke, da bakterien stadig er i gang med at tilpasse sig til næringsmediet. Mængden af tid denne fase tager varier meget og viser behovet for produktion af nye enzymer som tillader bakteriet at udnytte næringsmediet. \\ \hline

    Eksponentiel fase &  I den eksponentielle vækstfase vil mængden af bakterier fordoble efter den samme mængde tid. Det betyder at hvis man startede med 20 bakterier og efter 5 min havde 40 bakterier så ville man vide at mængden af bakterier altid vil fordoble efter 5 min. \\ \hline

    Stationær fase &  I den stationære fase vil mængden af bakterier ikke stige, da mængden af nye bakterier modsvares af mængden af bakterier. \\  \hline 

    Dødsfase & I dødsfasen vil mængden af bakterier der dør bliver større end mængden af nye baktierer hvilket betyder at mængden af totalle bakterier vil falde. \\ \hline
\end{longtable}
Mitose og meiose er de to forskellige måder en celle kan dele sig. Dog er det ikke tilfældigt hvilken af de to måder cellen kommer til at dele sig på. Hvis cellen er en kønscelle vil den gå igennem meiose for at dele sig og hvis cellen ikke er en kønscelle vil den gå gennem mitose for at dele sig.
I mitosen er der fem forskellige faser som en celle går igennem før den bliver delt og har lavet flere celler. Interfasen, profasen, metafasen, anafasen og telofasen. Formålet for mitosen er fomering, vækst og vedligeholdelse.

\begin{longtable}{| m{3cm} | m{14cm} |}
    \hline 
    Fase & Beskrivelse \\ \hline
    Interfasen & Interfasen er den fase som cellen bruger det meste af tiden i, da det er den fase hvor cellen bare laver de normalle ting en celle gør. Dog når cellen kommer tættere på at dele sig selv begynder DNA-strengene (kromatin) at strække sig ud, centrosomer og DNA begynder også at kopierer sig selv. \\ \hline

    Profasen & I profasen begynder kromatinet at kondensere og begynder at ligne kromosomer. Disse kromosomer består af to identiske søsterkromatider, som er forbundet i midten ved et centromer. Cellekernen, som indeholder alt af en celles DNA, begynder også at blive nedbrudt i denne fase og centrosomerne flytter sig hen til forskellige sider af cellen \\ \hline

    Metafasen & I metafasen, nu hvor kromosomerne er fuldt kondenseret, begynder kromosomerne at flytte sig ind til midten af cellen, for at danne en lige linje. De danner en lige linje for at sikre sig at søsterkromatiderne kan blive skilt ordenligt i den næste fase af mitosen. \\ \hline

    Anafasen & Under anafasen bliver kromosomerne skilt og de to søsterkromatider som kromosomerne bestod af bliver trukket til hver ende af cellen. Selve cellen begynder også at strække sig og cellen bliver tættere på at dele sig helt. \\ \hline

    Telofasen & Under telofasen bliver selve cellen splittet og bliver lavet om til to identiske datterceller. Kromosomerne begynder også at returnere til deres tidligere form som kromatin og forskellige organeller i cellen, ligesom cellekernen, begynder at blive genformet. \\ \hline
    
\end{longtable}

Meiosen er en måde at en celle kan dele sig, dog er det kun kønsceller eller gameter, som går igennem meiosen. Der er ni faser i meiosen, som er interfasen, profasen I, metafasen I, anafasen I, telofasen I, profasen II, metafasen II, anafasen II og telofasen II. Formålet med meiose er at danne kønsceller i forbindelse med kønnet formering.
\begin{longtable}{ | m{3cm} | m{14cm} |}
    \caption{Meiose} \\
    \hline
    Fase & Beskrivelse \\ \hline
    
    Interfase & Interfasen under meiose er præcist den samme som under mitosen. Cellen begynder at kopierer dens DNA og  gør sig klar til at begynde processen. \\ \hline

    Profase & Den første gang cellen går igennem profasen ligner fasen meget profasen i mitose, da kromatinet begynder at kondensere og begynder også at ligne kromosomer. Dog bliver kromosomernes genetiske materiale udvekles med andre kromosomer. Det gør så cellen opår genetisk variation. \\ \hline

    Metafase &  I den første metafase blier kromosomerne arrageret i par og bliver fastgjort til spindelapparatet for at gøre dem klar til at blive trukket til begge sider af cellen. \\ \hline

    Anafase & I anafase I bliver kromosomparrene delt og trukket til begge sider. Dog ligesom i mitosen bliver kromosomerne skilt og de to kromatider som kromosomerne bestod af bliver også trukket til hver side af cellen. \\ \hline

    Telofase & I telofase I bliver cellerne delt helt, hvilket betyder at der nu er to datterceller som er haploide, hvilket betyder at deres kromosomantal er halveret. Nye cellekerne bliver også dannet og kromosomerne begynder at dekondensere. \\ \hline

   Profase II & Igen i profase II begynder kromosomerne at blive kondenseret og cellekernet bliver også nedbrudt igen. Centriolerne begynder også at flytte sig hen til modsatte sider af cellen for at gøre klar til de næste faser. \\ \hline
 
    Metafase II &  Kromosomerne begynder at bevæge sig ind mod midten af cellen i en lige linje og kromosomerne gør sig klar til at blive adskilt i den næste fase. \\ \hline

    Anafase II & Kromosomerne begynder at bevæge sig ind mod midten af cellen i en lige linje og kromosomerne gør sig klar til at blive adskilt i den næste fase. \\ \hline

    Telofase II & I telofase II bliver cellerne delt igen hvilket betyder at der nu er fire celler til sidst. Nye cellekerne bliver også genbygget inde i cellerne for at holde på en celles DNA. Kromosomerne dekondenserer også igen og meiosen er færdig efter dette trin. \\ \hline

\end{longtable}

\begin{longtable}{ | m{5cm} | m{5cm} | m{5cm} | }
    \hline 
    & Meiose & Mitose \\ \hline
    Genetisk variation & Ja & Nej \\ \hline
    Antal datterceller & 4 & 2 \\ \hline
    Type celle & kønsceller & Somatiske celler \\ \hline
    mængde divisioner & 2 & 1 \\ \hline
\end{longtable}

I forsøget om osmotisk salinitet skulle vi putte lige store stykker kartofler ind i reagensglas med vand med forskellige koncentrationer af salt. Vi fandt ud af når kartoflen kom ind i et hypotonisk miljø altså et miljø med højere vandkonkenctration end inde i kartoflen ville massen af kartoflen stige, da vandet vil bevæge sig ind i cellerne. Vi fandt også ud af når kartoflen var i et hypertonisk miljø, altså et miljø hvor koncentrationen af ikke flytbart stof er højere end inde i kartoflen, ville massen af kartoflen formindske, da vandet vil bevæge sig ud af cellerne.
Salt kan derfor godt blive brugt som et konserveringsmiddel da ved brug af osmose vil det danne et hypertonisk miljø. Et miljø hvor stofkoncentrationen er højere udenfor cellerne end inde i cellerne, og vandet vil trænge gennem en semipermeabel membran for at udligne miljøet og opnå et isotonisk miljø. Salt virker derfor godt da det vil få vandet i bakterier til at gå gennem membranen og derfor også dræbe bakterien, da de har brug for vand til at trive. Salt vil også stoppe mug og skimmelsvamp da det også mængden af vand inde i cellerne vil falde. Det betyder at mad vil kunne være spiseligt i meget længere tid.

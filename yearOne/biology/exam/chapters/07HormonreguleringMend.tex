\newpage
\part{Hormonregulering hos mænd}
\subsection*{Forklar opbygningen og funktionen af de mandlige kønsorganer}
\subsection*{Redegør for hvad hormoner er samt hvordan de transporteres og reguleres i kroppen.}
\subsection*{I forhold til øvelsen om prævention og sexuelt overførte sygdomme ønskes en redegørelse for præventions virkemåde. Kom desuden ind på fejlkilder i forsøget}Både mænd og kvinder har mange forskellige kønsorganer med forskellige funktioner. Hos mænd har man 11 forskellige kønsorganer, sædlederen, kønsben, sædblæren, blærehalskirtlen, cowpersk kirtel, svulmlegemer, urinrør, bitestikel, penishoved, testikel og pungen. Dog er det primære kønsorgan for mænd testiklerne. I testiklerne bliver sædcellerne produceret som er nødvendige for at få afkom. Bag testiklerne sidder bitestiklerne hvor sædcellerne bliver opbevaret og modnet. Både testiklerne og bitestiklerne sidder i pungen som er 2-4° koldere end resten af kroppen for at sædcellerne kan udvikle sig normalt. Sædcellerne føres op fra bitestiklen gennem sædlederen for at blande sæden med en væske fra sædblæren. Denne væske indholder det sædcellerne bruger som energikilde til deres bevægelser, altså fruktose. Sæden får også mere væske fra blærehalskirtlen. Før sæden bliver udløst tilføjes endnu en væske. Væsken tilføjes fra cowpers kirtel og er basisk så sædcellerne kan overleve i det sure miljø i kvindens skede. Udløsningen af sæden er gennem urinrøret, som går gennem helle penissen og også penishovedet. Penishovedet sider yderst på penis og er tæt besat med nerveceller for at få manden til at ejakulere hurtigere. Når manden er seksuelt påvirket, rejser penissen, det gør den da svulmelegemerne i penissen fyldes med blod og de blodfyldte legemer klemmer på de vener der skal føre blodet væk. Penissen ved at den skal rejse når den modtager nervesignaler fra hjernen og rejsningen sker spontant. Efter penissen  har rejst bliver signalerne nedbrudt, og arterierne trækker sig sammen. 
Når en mand er blevet kønsmoden bliver han ved med at producere sædceller resten af livet. Hver dag bliver cirka 100-200 millioner sædceller lavet, dog når man bliver ældre falder mængden. En udløsning af sæd kan indeholde op til 300 millioner sædceller. Sædcellens hoved er fuldt med tætpakket DNA og enzymer som cellen skal bruge for at trænge ind i ægcellen. Selve sædcellen indeholder 22 autosomer og er X- eller Y-kromosom. Bag hovedet sider et mellemstykke som har en masse mitokondrier. De bruges når cellen skal have energi til at bevæge sig gennem kvindens skede og æggeleder. Sædceller kan bevæge sig ved at piske den hale fra side til side. Produktionen af sædceller sker i sædkanalerne som findes i testiklerne.

Hormoner er stoffer der er nødvendige for en organisme, da de lader celler kommunikere med hinanden. Hormoner er som sagt stoffer, som bliver udskillet af specielle celler, og bliver brugt til at kommunikere med andre celler, der kan genkende hormonet. Celler kan genkende hormoner ved brug af hormon-receptorer. Hormoner bliver typisk lavet af endokrine kirtler, og hormoner bliver udskillet til blodbanen. Hos både mænd og kvinder bliver kønshormonerne primært udskillet fra kønskirtlerne, og selve reguleringen af hormonerne foregår mellem hypothalamus, hypofysen og kønskirtlerne. 
Det primære hormon hos mænd er testosteron, som bliver dannet af et negativt feedback loop. I hypothalamusen bliver hormonet GnRH dannet, hvilket får hypofysen til at danne to andre kønshormoner, FSH og LH. Ved brug af blodet bliver disse to hormoner føret hen til testiklerne for at danne testosteron. Dog skal koncentrationen af testoteron reguleres for kroppen holdes i balance. Når der bliver dannet for meget testosteron vil den høje koncentration stoppe produktionen af GnRH og LH. Det betyder at koncentrationen af GnRH og LH i kroppen mindskes, hvilket betyder at testosteronproduktionen stoppes og koncentrationen af testosteron formindskes. Det betyder så også at når koncentrationen af testosteron er faldet, så ville hypothalamusen begynde at producere GnRH og LH igen. 
Under puberten begynder testosteron at blive dannet og har stor indflydelse på en drengs krop. Testosteronen gør så sædceller begynder at blive dannet. Penissen og pungen begynder at vokse. Man får øget muskelmasse og kraftigere knogler. Man får også skæg, flere røde blodlegemer og acne.

I øvelsen om prævention og sexuelt overførte sygdomme skulle vi se hvordan kønssygdomme spreder sig og om præventionsmiddler hjælper med at stoppe spredningen af kønssygdomme. Vi brugt reagensglas som en erstatning af penissen. Kondom som vores præventionsmiddel, petriskåle med agar for at vi kunne tydeligt se hvor bakterierne spredte sig. Vi brugte også E.Coli som en erstatning af klamydia da det ikke ville have været sikkert for os at lege med en kønssygdom. Min gruppe kom dog til at lave fejl under forsøget da vores resultater gik helt imod alle andre gruppers. Den første fejl vi lavede var at vi pressede reagensglasset alt for hårdt mod agaren i petriskålene. Det gjorde så det var svært for os at aflæse resultaterne og har måske også revet hul i kondomen, hvilket er et selvfølge at det er en fejl, da forsøget handlede om præventionsmidler. Vi glemte også at tage stilling for at det kun var E.Colien der kom ind i petriskålene og vi regner med at der også har været noget kontaminering fra kimnedfald som har forvrænget vores resultater. 
Vi ved dog at en kondom burde fungere ved at sikre sig at der er ingen kontakt mellem det mandlige kønsorgan og andre steder. Det burde stoppe spredningen af kønssygdomme da det fungerer som en barriere og stoppe en overførelse af kropsvæsker mellem partnere. Dog er det jo ikke det eneste kondommer bliver brugt til da det også virker som en måde at forhindre gravidtitet, ved at, som sagt, forhindre at sæd bliver overført fra penissen til skeden.

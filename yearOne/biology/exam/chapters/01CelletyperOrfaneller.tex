\newpage
\part{Celletyper og deres organeller}\label{sec:celletyperogderesorganeller}
    \section*{Redegør for cellers opbygning, herunder forskelle og ligheder mellem forskellige celletyper.}
        For at kunne påpeje hvad forskellighederne mellem de forskellige celletyper er tror jeg at det er vigtit at man ved hvad de forskellige celle typer er. 
        Der finde 2 forskellige celletyper, som er:
        \begin{itemize}
            \item Prokaryote celler (Celler unden cellekernen)
            \item Eukaryote celler (Celler med Cellekernen)
        \end{itemize}
        Ganske kort så er forskellen på de 2 celletyper at prokaryote celler ikke har en cellekerne, hvorimod eukaryote celler har en cellekerne.
        \subsection*{Hvad er en Eukaryot celle?}
            Eukaryote celler kender vi fra utallige steder, da det er de celler som vi finder i planter og dyr, Herunder også mennesker.
            De eukaryote celler indeholder som sagt en cellekerne. De er forholdsvis store, og er afgrænset af en cellemembran. 
            Et overall navn for alt der er i en celle undtagen cellekerne er cytoplasma. Cytoplasma er den del af cellen som indeholder alle organellerne, såsom mitokondrier, ribosomer, og mange flere.
            \paragraph{Cellekernen (nucleus)}
                Cellekernen er en af de vigtigste organeller i en celle. I mennesket celler er der 46 kromosomer, og 23 kromosom-par. Hvert kromosom, indeholder et DNA-mollekyle og særlige proteiner.(Se mere her\ref{sec:protein}).
                Cellekernen, også kendt som nucleus, er en af de mest fundamentale komponenter i en celle. Den er en organell, som er omgivet af en dobbeltmembran og indeholder det meste af cellens genetiske materiale. Her er nogle af de vigtigste karakteristika ved cellekernen:

                Genetisk materiale: Cellekernen indeholder det meste af cellens DNA, som er organiseret i strukturer kendt som kromosomer. DNA indeholder genetisk information, der er nødvendig for at kontrollere cellefunktioner og arvelighed.

                DNA-replikation og transkription: Cellekernen er stedet for DNA-replikation (kopiering af DNA før celledeling) og transkription (processen, hvor en del af DNA-sekvensen kopieres til mRNA, som derefter bruges til at lave proteiner).

                Regulering af genekspression: Cellekernen spiller også en central rolle i reguleringen af genekspression, det vil sige processen, hvor information fra et gen bruges til at lave et funktionelt produkt, typisk et protein.

                Beskyttelse af genetisk materiale: Ved at indeholde DNA i en adskilt struktur, beskytter cellekernen DNA fra potentielle skader, der kunne forekomme i cytoplasma.

                Det er vigtigt at bemærke, at ikke alle celler har en cellekerne. For eksempel har bakterier ikke en cellekerne; i stedet er deres DNA fritflydende i cellen. Dette er en af de vigtigste forskelle mellem prokaryote (uden cellekerne) og eukaryote (med cellekerne) celler.
                
            \paragraph{Mitokondrier}
                En ting som prokaryote celler heller ikke har er Mitokondrier. Det har en eukaryote celle dog. Mitokondrier er et organell som står for at danne ATP.
                ATP er et molekyle som indeholder kemisk energi, som cellen kan bruge til at udføre arbejde. Behovet for ATP er meget stort, da det er med til at drive mange af cellernes processer, herunder celledeling.   
            
        \subsection*{forskelle og ligheder mellem planteceller og dyre celler}
            En eukaryote celle er som før nævnt en celle som har en cellekerne, og ses ved blandt andre Dyre celler og planteceller. Men hvad er forskellen på en dyre celle og en plantecelle, og hvad har de tilfældes?
            \newline\textbf{Ligheder mellem planteceller og dyreceller:}\newline
            Cellekernen: Begge har en cellekerne, der indeholder cellens DNA.
            Organeller: Begge indeholder organeller, såsom mitokondrier, endoplasmatisk reticulum, og Golgi-apparatet.

            Cytoplasma: Begge indeholder cytoplasma, en geléagtig substans, der fylder cellen og indeholder organellerne.

            Cellemembran: Begge har en cellemembran, der styrer, hvilke stoffer der kan komme ind og ud af cellen. \\
            \textbf{Forskelle mellem planteceller og dyreceller:}\newline

            Cellevæg: Planteceller har en cellevæg lavet af cellulose, der giver ekstra struktur og støtte. Dyreceller har ikke en cellevæg.

            Kloroplaster: Planteceller indeholder kloroplaster, hvor fotosyntese foregår for at producere glucose. Dyreceller har ikke kloroplaster, da de ikke udfører fotosyntese.

            Store central vakuole: Planteceller indeholder en stor central vakuole, der lagrer vand og hjælper med at opretholde celleturgor. Selvom dyreceller også kan have vakuoler, er de generelt mindre og ikke så fremtrædende som i planteceller.

            Cytoskelet og centrosomer: Dyreceller har centrosomer, der hjælper med celledeling, mens de fleste planteceller ikke har. Desuden er dyrecellers cytoskelet mere udtalt og komplekst end det hos planteceller.

    \subsection*{Redegør for cellernes forskellige organeller samt deres funktion. Kom herunder ind på fotosyntese og respiration.}
        Eukaryote celler, der er til stede i planter, dyr, svampe og protister, rummer en række specialiserede organeller, hver med sine unikke funktioner. Lad os dykke ned i nogle af de mest markante organeller og deres roller:

        I planteceller finder man organeller kaldet kloroplaster. Kloroplaster er ansvarlige for fotosyntesen, en proces, der konverterer lysenergi til kemisk energi. Da fotosyntese ikke forekommer i dyreceller, er disse celler uden kloroplaster.
        
        Både planteceller og dyreceller indeholder mitokondrier, som er centrale for produktionen af ATP. ATP, eller adenosintrifosfat, er et molekyle, der indeholder kemisk energi, som cellen kan bruge til at drive forskellige processer. ATP's rolle er vital, idet det driver mange af cellens processer, herunder celledeling.
        
        Men hvordan produceres ATP? Svaret ligger i en proces kaldet respiration. Respiration er en biologisk proces, hvor celler nedbryder glukose for at frigive energi, som de kan bruge. Der er to hovedtyper af respiration: aerob og anaerob.
        
        Aerob respiration, der kræver ilt, er en mere effektiv proces sammenlignet med anaerob respiration. Det er fordi aerob respiration producerer omkring 36 ATP-molekyler, mens anaerob respiration, der kan forekomme uden ilt, kun producerer 2 ATP-molekyler. En anden forskel er, at anaerob respiration resulterer i produktionen af mælkesyre, mens aerob respiration ikke gør det. Desuden forekommer anaerob respiration i cytoplasmaet, mens aerob respiration hovedsageligt forekommer i mitokondrierne. \\
        \textbf{Cellulær Respiration: (Cellulær respiration kan være både aerob og anaerob) }\begin{equation}C_6H_{12}O_6 + 6CO_2 \rightarrow 6O_2 + 6H_2O + ATP \end{equation}


    \subsection*{Diskuter forsøget om papirchromatografi, og kom i den forbindelse ind på betydningen af fotosyntese og respiration, i organismen, såvel som i et økosystem}
        % TODO: Skriv om papirchromatografi
        

    \subsection*{Yderlige informationer om Celler og organeller}
    Hvordan er det nu med de proteiner i DNA-mollekylet? \label{sec:protein} En DNA-streng er indpakket omkring otte histonproteiner for at danne en struktur kaldet en "nukleosom". En serie af nukleosomer, der ligner perler på en snor, snoes og foldes for at danne en mere kompleks struktur kaldet kromatin. 

\newpage
\part{Celletyper og deres organeller}\label{sec:celletyperogderesorganeller}
    \section{Redegør for cellers opbygning, herunder forskelle og ligheder mellem forskellige celletyper.}
        For at kunne påpeje hvad forskellighederne mellem de forskellige celletyper er tror jeg at det er vigtit at man ved hvad de forskellige celle typer er. 
        Der finde 2 forskellige celletyper, som er:
        \begin{itemize}
            \item Prokaryote celler (Celler unden cellekernen)
            \item Eukaryote celler (Dyre celle og planteceller)
        \end{itemize}
        Ganske kort så er forskellen på de 2 celletyper at prokaryote celler ikke har en cellekerne, hvorimod eukaryote celler har en cellekerne.
        \subsection{Hvad er en Eukaryot celle?}
            Eukaryote celler kender vi fra utallige steder, da de er de celler som vi finder i planter og dyr. Herunder mennesker.
            De eukaryote celler indeholder som sagt en cellekerne. De er forholdsvis store, og er afgrænset af en cellemembran (cellemembranen). 
            Et overall navn for alt der er i en celle undtagen cellekerne er cytoplasma. Cytoplasma er den del af cellen som indeholder alle organellerne, såsom mitokondrier, ribosomer, og mange flere.
            \paragraph{Cellekernen (nucleus)}
                Cellekernen er en af de vigtigste organeller i en celle. I mennesket celler er der 46 kromosomer, og 23 kromosom-par. Hvert kromosom, indeholder et DNA-mollekyle og særlige proteiner.(Se mere her\ref{sec:protein}).
                Cellekernen er omgivet af en dobbelt membran, som er med til at beskytte DNA'et. 
                Cellekernen indeholder også en cellekernemembran, som er med til at beskytte DNA'et mod skader fra cytoplasmaet.
                Cellekernen indeholder også en cellekerneporer, som er med til at transportere RNA ud af cellekernen.
                Cellekernen indeholder også en cellekernenukleolus, som er med til at producere ribosomer.
            \paragraph{Mitokondrier}
                En ting som prokaryote celler heller ikke har er Mitokondrier, det har en eukaryote celle dog. Mitokondrier er et organell som står for at danne ATP.
                ATP er et molekyle som indegolder kemisk energi, som cellen kan bruge til at udføre arbejde. Behovet for ATP er meget stort, da det er med til at drive mange af cellernes processer, herunder celledeling.   
            
        \subsection{forskelle og ligheder mellem planteceller og dyre celler}
            En eukaryote celle er som før nævnt en celle som har en cellekerne, og ses ved blandt andre Dyre celler og planteceller. Men hvad er forskellen på en dyre celle og en plantecelle, og hvad har de tilfældes?
            \newline\textbf{Ligheder mellem planteceller og dyreceller:}\newline
            Cellekernen: Begge har en cellekerne, der indeholder cellens DNA.
            Organeller: Begge indeholder organeller, såsom mitokondrier, endoplasmatisk reticulum, og Golgi-apparatet.

            Cytoplasma: Begge indeholder cytoplasma, en geléagtig substans, der fylder cellen og indeholder organellerne.

            Cellemembran: Begge har en cellemembran, der styrer, hvilke stoffer der kan komme ind og ud af cellen. \\
            \textbf{Forskelle mellem planteceller og dyreceller:}\newline

            Cellevæg: Planteceller har en cellevæg lavet af cellulose, der giver ekstra struktur og støtte. Dyreceller har ikke en cellevæg.

            Kloroplaster: Planteceller indeholder kloroplaster, hvor fotosyntese foregår for at producere glukose. Dyreceller har ikke kloroplaster, da de ikke udfører fotosyntese.

            Store central vakuole: Planteceller indeholder en stor central vakuole, der lagrer vand og hjælper med at opretholde celleturgor. Selvom dyreceller også kan have vakuoler, er de generelt mindre og ikke så fremtrædende som i planteceller.

            Cytoskelet og centrosomer: Dyreceller har centrosomer, der hjælper med celledeling, mens de fleste planteceller ikke har. Desuden er dyrecellers cytoskelet mere udtalt og komplekst end det hos planteceller.

    \subsection{Redegør for cellernes forskellige organeller samt deres funktion. Kom herunder ind på fotosyntese og respiration.}
    Eukaryote celler, som findes i planter, dyr, svampe og protister, indeholder forskellige organeller, som hver har unikke funktioner. Nedenfor er nogle af de mest bemærkelsesværdige organeller og deres funktioner:
    I en plantecelle vil man se grønkorn. I gørnkornene foregår fotosyntesen, som bekendt sker der ikke fotosyntese i dyreceller, og derfor er der ikke grønkorn i dyreceller. Ud over det har man i plante celler og dyre celler mitokondrier, som er med til at danne ATP. ATP er et molekyle som indegolder kemisk energi, som cellen kan bruge til at udføre arbejde. Behovet for ATP er meget stort, da det er med til at drive mange af cellernes processer, herunder celledeling.
    Hvad er respiration for en process? Respiration er en proces, hvor celler nedbryder glukose til at frigive energi, som de kan bruge. Respiration er en af de vigtigste måder, hvorpå celler frigiver energi fra glukose. Der er to typer respiration: aerob og anaerob. 
    Aerob respiration kræver ilt, mens anaerob respiration ikke gør det. Aerob respiration er mere effektiv end anaerob respiration, da det producerer mere ATP. Anaerob respiration producerer kun 2 ATP-molekyler, mens aerob respiration producerer 36 ATP-molekyler. Anaerob respiration producerer også mælkesyre, mens aerob respiration ikke gør det. Anaerob respiration forekommer i cytoplasmaet, mens aerob respiration forekommer i mitokondrierne. \\
    \textbf{Cellulær Respiration: (Cellulær respiration kan være både aerob og anaerob) }\begin{equation}C_6H_{12}O_6 + 6CO_2 \rightarrow 6O_2 + 6H_2O + ATP \end{equation}


    \subsection{Diskuter forsøget om papirchromatografi, og kom i den forbindelse ind på betydningen af fotosyntese og respiration, i organismen, såvel som i et økosystem}
        % TODO: Skriv om papirchromatografi

    \subsection{Yderlige informationer om Celler og organeller}
    Hvordan er det nu med de proteiner i DNA-mollekylet? \label{sec:protein} En DNA-streng er indpakket omkring otte histonproteiner for at danne en struktur kaldet en "nukleosom". En serie af nukleosomer, der ligner perler på en snor, snoes og foldes for at danne en mere kompleks struktur kaldet kromatin. 

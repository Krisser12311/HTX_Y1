\newpage
\section{Transport ind og ud af celler}
    \subsection{Redegør for opbygningen af celler. Herunder deres organeller samt transport ind og ud af cellen.}
        For at læse omkring celler henvises til \textbf{Kapitel 1: Celler} på side \pageref{sec:celletyperogderesorganeller}.
        \begin{center}
            \begin{longtable}{ | m{2cm} | m{11cm}| m{2cm} |}
                \caption{Transport-processer} \\
                \hline
                \textbf{Transport-proces} & \textbf{Hvordan fungere processen og hvilke stoffer transporteres på denne måde?} & \textbf{Aktiv eller pasivt}\\
                \hline
                Diffusion & Diffusion er en proces, der sker naturligt i naturen, hvor molekyler bevæger sig fra områder med høj koncentration til områder med lav koncentration. 
                Denne proces fortsætter, indtil der er ligevægt, det vil sige, at koncentrationen af molekyler er den samme overalt. Der er to hovedtyper af diffusion, der finder sted i celler: 
                simpel diffusion og faciliteret diffusion. \textbf{Simpel diffusion:}  ette er den grundlæggende form for diffusion, hvor molekylerne bevæger sig frit gennem cellemembranen uden brug af transportproteiner. 
                De molekyler, der typisk bevæger sig på denne måde, er små, ikke-polære molekyler, såsom oxygen og carbon dioxide. Vand kan også passere gennem membranen på denne måde via en proces kaldet osmose. Det kræver ingen energi (ATP) fra cellen, og derfor betegnes det som en passiv transportform. & Passiv\\
                \hline
                Osmose & Osmose er en specifik type af passiv transport, der er meget vigtig i biologiske systemer. Den involverer bevægelsen af vandmolekyler fra et område med lav solutkoncentration (høj vandkoncentration) til et område med høj solutkoncentration (lav vandkoncentration) gennem en semipermeabel membran.

                En semipermeabel membran er en type barriere, der tillader visse stoffer at passere igennem, men blokerer for andre. I tilfældet med osmose tillader den vand at passere igennem, men forhindrer mange andre molekyler, særlig store eller ladede molekyler, i at gøre det.
                
                Der er tre hovedtyper af osmotiske forhold:
                \textbf{Isotonisk: }  Her er koncentrationen af solut (opløst stof) den samme på begge sider af membranen. Der vil derfor ikke være nogen bevægelse af vand. 
                \textbf{Hypertonisk: } Her er koncentrationen af opløst stof højere uden for cellen end inden i den.
                Vand vil tendere til at bevæge sig ud af cellen for at fortynde soluten uden for cellen, hvilket kan forårsage cellen til at skrumpe.
                \textbf{Hypertonisk: } er er koncentrationen af opløst stof lavere uden for cellen end inden i den. Vand vil tendere til at bevæge sig ind i cellen for at fortynde oplæste stof inde i cellen, hvilket kan forårsage cellen til at svulme og potentielt briste. & Passiv\\
                \hline
                Aktiv transport & Aktiv transport er en proces hvor molekyler bevæger sig fra et område med lav koncentration til et område med høj koncentration. Større molekyler kan komme igemmen her. & Aktiv\\
                \hline
                Faciliteret diffusion & Faciliteret diffusion er en proces hvor molekyler bevæger sig fra et område med lav koncentration til et område med høj koncentration. Glukose kan blandes med vand og derfor være et polært stof. Og det har derfor ikke diffundere ved brug af simpel diffusion hen over en celle membran, da en cellemembran er lavet af Fosfolipider altså fedt stof. Derfor skal glukose bruge en transport protein for at komme igennem cellemembranen.  
                & Passiv\\
                \hline
            \end{longtable}
        \end{center}

    \subsection{Forklar forsøget: Osmose i kartofler.}
    \subsection{Diskuter hvordan transport over cellemembranen spiller en vigtig rolle for organismer.}
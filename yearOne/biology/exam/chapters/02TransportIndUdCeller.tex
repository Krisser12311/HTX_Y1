\newpage
\part{Transport ind og ud af celler}
    \section*{Redegør for opbygningen af celler. Herunder deres organeller samt transport ind og ud af cellen.}
        For at læse omkring celler henvises til \textbf{Kapitel 1: Celler} på side \pageref{sec:celletyperogderesorganeller}.
        \begin{center}
            \begin{longtable}{ | m{2cm} | m{11cm}| m{2cm} |}
                \caption{Transport-processer} \\
                \hline
                \textbf{Transport-proces} & \textbf{Hvordan fungere processen og hvilke stoffer transporteres på denne måde?} & \textbf{Aktiv eller pasivt}\\
                \hline
                Diffusion & Diffusion er en proces, der sker naturligt i naturen, hvor molekyler bevæger sig fra områder med høj koncentration til områder med lav koncentration. 
                Denne proces fortsætter, indtil der er ligevægt, det vil sige, at koncentrationen af molekyler er den samme overalt. Der er to hovedtyper af diffusion, der finder sted i celler: 
                simpel diffusion og faciliteret diffusion. \textbf{Simpel diffusion:}  ette er den grundlæggende form for diffusion, hvor molekylerne bevæger sig frit gennem cellemembranen uden brug af transportproteiner. 
                De molekyler, der typisk bevæger sig på denne måde, er små, ikke-polære molekyler, såsom oxygen og carbon dioxide. Vand kan også passere gennem membranen på denne måde via en proces kaldet osmose. Det kræver ingen energi (ATP) fra cellen, og derfor betegnes det som en passiv transportform. & Passiv\\
                \hline
                Osmose & Osmose er en specifik type af passiv transport, der er meget vigtig i biologiske systemer. Den involverer bevægelsen af vandmolekyler fra et område med lav solutkoncentration (høj vandkoncentration) til et område med høj solutkoncentration (lav vandkoncentration) gennem en semipermeabel membran.

                En semipermeabel membran er en type barriere, der tillader visse stoffer at passere igennem, men blokerer for andre. I tilfældet med osmose tillader den vand at passere igennem, men forhindrer mange andre molekyler, særlig store eller ladede molekyler, i at gøre det.
                
                Der er tre hovedtyper af osmotiske forhold:
                \textbf{Isotonisk: }  Her er koncentrationen af solut (opløst stof) den samme på begge sider af membranen. Der vil derfor ikke være nogen bevægelse af vand. 
                \textbf{Hypertonisk: } Her er koncentrationen af opløst stof højere uden for cellen end inden i den.
                Vand vil tendere til at bevæge sig ud af cellen for at fortynde soluten uden for cellen, hvilket kan forårsage cellen til at skrumpe.
                \textbf{Hypertonisk: } er er koncentrationen af opløst stof lavere uden for cellen end inden i den. Vand vil tendere til at bevæge sig ind i cellen for at fortynde oplæste stof inde i cellen, hvilket kan forårsage cellen til at svulme og potentielt briste. & Passiv\\
                \hline
                Aktiv transport & Aktiv transport er en proces hvor molekyler bevæger sig fra et område med lav koncentration til et område med høj koncentration. Større molekyler kan komme igemmen her. & Aktiv\\
                \hline
                Faciliteret diffusion & Faciliteret diffusion er en proces hvor molekyler bevæger sig fra et område med lav koncentration til et område med høj koncentration. glucose kan blandes med vand og derfor være et polært stof. Og det har derfor ikke diffundere ved brug af simpel diffusion hen over en celle membran, da en cellemembran er lavet af Fosfolipider altså fedt stof. Derfor skal glucose bruge en transport protein for at komme igennem cellemembranen.  
                & Passiv\\
                \hline
            \end{longtable}
            \begin{longtable}{ | m{2cm} | m{5cm}| m{2cm} | m{3cm} | m{3cm} |}
                \caption{Transport metoder} \\
                \hline
                Transport metode & \textbf{Definition} & \textbf{Hvilke stoffer} & \textbf{Forudsætninger} & \textbf{Eksempler} \\
                \hline
                Simpel diffusion & Diffusion gennem membranens fedthinde eller proteinkanaler i membranen & Små upolære molekyler & koncentrations gradient\newline Opløslighed i fedt & Ilt \begin{math}O_2\end{math} \newline Kuldioxid \begin{math}CO_2\end{math} \\
                \hline
                Osmose \newline Simpel diffusion af \begin{math}H_2O\end{math} & Diffusion af vand gennem membranens fedthinde eller proteinkanaler i membranen & Vand & koncentrations gradient\newline Lille molekyle størelse & Vand \begin{math}H_2O\end{math} \\
                \hline
                Faciliteret diffusion & Diffusion gennem transportprotein i membranen & Større polære molekyler og ioner & koncentrations gradient\newline Specefikt transportprotein & glucose \begin{math}C_6H_{12}O_6\end{math}, Natrium \begin{math}Na^+\end{math}, Kalium \begin{math}K^+\end{math} \\
                \hline
                Aktiv transport & Transport gennem transportprotein som kun fungere sammen med ATP & Polære stoffer og ioner & ATP og Specefikt transportprotein & \begin{math}Na^+\end{math}  \newline \begin{math}K^+\end{math}  \newline \begin{math}NO_3^-\end{math}  \newline \begin{math}PO_4^{3-}\end{math} \newline \begin{math}C_6H_{12}O_6\end{math} \\
                \hline
            \end{longtable}
        \end{center}

    \section*{Forklar forsøget: Osmose i kartofler.}
        Selve opgaven i forsøget var at finde frem til hvad type vand en person var druknet i. Det kan man finde frem til på flere forskellige metoder. Det vi gjorde var at vi tog nogle kendte opløsninger af saltvand og lagde en del af en katoffel ned i den. Ved at veje kartoffelstykkerne før og efter de har været i de kendte saltvandsopløsninger, kan man bestemme, hvordan saltkoncentrationen påvirker vægtændringen. Ved at plotte disse data og lave en regressionsanalyse, kan man så bestemme saltkoncentrationen i den ukendte opløsning ved at se på, hvordan den påvirkede vægten af et kartoffelstykke.. Når man lægger et stykke kartoffel i en opløsning, vil vandet enten bevæge sig ind i kartoffelcellerne eller ud af dem afhængigt af koncentrationen af salt i opløsningen sammenlignet med koncentrationen af salt inde i cellerne. Hvis opløsningen er hypertonisk (har en højere saltkoncentration end kartoffelcellerne), vil vandet bevæge sig ud af cellerne for at forsøge at udligne koncentrationerne, og kartoffelstykket vil tabe vægt. Hvis opløsningen er hypotonisk (har en lavere saltkoncentration end kartoffelcellerne), vil vandet bevæge sig ind i cellerne, og kartoffelstykket vil tage på i vægt.
    
        \section*{Diskuter hvordan transport over cellemembranen spiller en vigtig rolle for organismer.} % TODO
            Diffusion: Diffusion er fundamentalt for udvekslingen af  gas i lungerne. Når vi indånder, har luften i vores lunger en højere koncentration af ilt end blodet i de omkringliggende blodkar. Dette skaber en koncentrationsgradient, som fører til, at ilt diffunderer ind i blodet, hvor det kan blive transporteret til resten af kroppen. På samme tid har blodet en højere koncentration af kuldioxid end luften i lungerne, hvilket fører til, at kuldioxid diffunderer ud af blodet og ind i lungerne, hvor det kan blive udåndet.

            Osmose: Osmose er afgørende for vandbalance i organismer. For eksempel, hvis en person drikker saltvand, vil det høje saltindhold i tarmene trække vand ud af kroppens celler og ind i tarmene ved osmose, hvilket fører til dehydrering. Dette er grunden til, at drikkevand er afgørende for vores overlevelse, da det hjælper med at opretholde den rigtige balance af vand inden for vores celler og forhindre dehydrering.

            Aktiv transport: Aktiv transport er afgørende for at opretholde elektrokemiske gradienter, som er afgørende for nervefunktion. Natrium-kalium-pumpen (Natrium-kalium-pumpens funktion er, at pumpe natrium- og kaliumioner ud af og ind i cellerne. Hver gang der pumpes 3 Na+ ud, pumpes der 2 K+ ind ved forbrug af et ATP. ), som er et eksempel på aktiv transport, bruger ATP til at pumpe natrium og kalium ioner imod deres koncentrationsgradienter. Dette skaber en elektrisk ladning over cellemembranen, som er nødvendig for transmissionen af nerveimpulser. Uden denne aktive transportproces ville vores nervesystem ikke fungere korrekt.

            Faciliteret diffusion: Faciliteret diffusion er vigtig for at sikre, at cellerne i vores krop kan absorbere glucose fra blodet. Glucose er et stort molekyle, der ikke kan diffundere gennem cellemembranen alene, så det kræver et transportprotein. Uden faciliteret diffusion ville vores celler ikke være i stand til at tage den glucose, de har brug for at producere energi.

            Samlet set spiller disse transportmekanismer en afgørende rolle for en organisms overlevelse, da de sikrer, at nødvendige molekyler og ioner kan bevæge sig ind og ud af cellerne på en kontrolleret måde.
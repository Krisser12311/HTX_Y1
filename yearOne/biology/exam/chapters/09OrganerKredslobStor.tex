\newpage
\part{Organer og kredsløb}
\subsection*{Forklar kort om nogle udvalgte organsystemer. Kom ind på deres funktion}
\subsection*{Redegør for kredsløbets opbygning og funktion, med særligt fokus på det store kredsløb}
\subsection*{I forhold til øvelsen: fysiologiske målinger (puls, blodtryk, fedtprocent, BMI og vitalkapacitet) ønskes en diskussion om hvordan motion spiller en rolle for sundhed.}
En organisme kan opdeles i en masse forskellige organsystemer, der har forskellige formål, integumentære system, skelettet, muskelsystemet, nervesystemet, endokrinsystemet, cardiovaskulære system, lymfesystemet, respirationssystemet, fordøjelsessystemet, urinvejssystemet, immunsystem og forplantningssystemet. 
Nervesystemet er det system der står for at regulere og koordinere forskellige aktiviter i kroppen. Selve systemet består af tre mindre systemer: Det centrale system, det perifere system og det autonome nervesystem. Det centrale nervesystem består af rygmarven og hjernen, begge af dem er lavet af nerveceller og nervefibre. Det centrale nervesystem står for at sende nerveimpulser fra hjernen hen til rygmarven hvor det så bliver sendt videre til det perifere system. Det perifere system er nerverne som fører impulserne fra hjernen og rygmarven hen til resten af kroppen. Det står også for at få sendt impulser fra resten af kroppen til rygmarven og derefter til hjernen så det er muligt for hjernen at kommunikere med musklerne. Det autonome nervesystem er et system som vi ikke kan kontrollerer bevidst. Det står for at kontrollerer kropstemperatur, puls, indre organer osv. 
Fordøjelsessystemet er det system der står for at indtage, transportere, nedbryde og optage livsvigtige stoffer. Selve fordøjelsessystemet kan ses som et 6-7 meter langt rør der går fra munden til endetarmen, og bliver kaldt fordøjelseskanalen. Fordøjelseskanalen består af mange forskellige afsnit altså, munden, spiserøret, mavesækken, tyndtarmen, tyktarmen og endetarmen. I munden bliver føden forabejdret af tænderne, som foretager en mekanisk bearbejdning af føden, og af spytkirtlerne som udskiller spytamylase, som nedbryder amylose til kortkædede kulhydrater. I mavesækken bliver proteiner brudt ned til kortkædede peptider og hele processen bliver katalyseret af pepsin. I tyndtarmen bliver kulhydrater spaltet til monosacchardier, proteiner bliver spaltet til aminosyrer og fedt bliver spaltet til glycerol og fedtsyrer. I tyktarmen bliver vand og salte optages, og ufordøjelige stoffer (cellulose) bliver forgæret af colibakterier. Til sidst i endetarmen bliver de ufordøjelige madrester, tarmbakterier og andre døde celler sendt ud i form af afføring.
Det endokrine system er det system der står for at lade cellerne i kroppen kommunikere med hinanden. Måden cellerne kan kommunikere med hinanden er ved brug af hormoner. Hormoner er stoffer som bliver udskilt til blodkredsløbet af kirtelceller, så hormonerne kan transporteres rundt i hele kroppen med blodet, dog er det kun celler med en specifik hormonreceptor som kan opfange specifikke hormoner. Hormoner bliver typisk brugt til at regulere udskillelsen af andre stoffer i kroppen der påvirker funktioner i kroppen. 

En af de vigtigere organsystemer i kroppen er blodkredsløbet. Det består af hjertet, blodkarrerne og blodet, og har til formål at transportere blod rundt for at få alle celler forsynet med ilt og næringsstoffer. Det bliver også brugt til at fjerne kuldioxid og andre affaldsstoffer. Typisk bliver kredsløbet opdelt i to dele, det lille kredsløb og det store kredsløb. Kort sagt kan man sige at i det lille kredsløb bliver blod sendt fra hjertet hen til lungerne og så tilbage til hjertet igen. I det store kredsløb bliver blodet sendt fra hjertet hen til resten af kroppen og så til sidst tilbage til hjertet igen.
I det store kredsløb bliver de iltede røde blodlegemer sendt ud fra den venstre ventrikel, hvor blodet går gen til aorta, som er kroppens største arterie, og bliver derfra forgrenet ud i hele kroppen. Blodet når rundt i hele kroppen for at samle affaldsprodukter fra celler, og til at give ilt til celler. Det bliver gjort ved brug af en process der hedder diffusion. Det går ud på at partikler flytter sig fra et område med høj koncentration til et område med lav koncentration ved brug af varmebevægelser. Efter de røde blodlegemer har været en tur rundt i kroppen kommer det tilbage til hjertet ved brug af højre atrium.

I øvelsen fysiologiske målinger skulle vi måle vores puls, blodtryk, fedtprocent, BMI og vitalkapacitet. Puls er mængden af gange hjertet slår hvert minut og ligger typisk ved de cirka 60-70 slag per minut. Blodtryk er trykket i blodet og ligger normalt på de cirka 120-135/70-85. Det første tal viser trykket i arterierne når hjertet trækker sig sammen og det andet tal er trykket mellem hjerteslagene. Fedtprocent er mængden af kroppen der er fedt i procent og ligger normalt på 8-20%. BMI eller body mass index er et tal der viser om man er overvægtig eller undervægtig baseret på ens højde og vægt. Det ligger typisk på 18,5 til 24,9. Til sidst skulle vi også måle vitalkapacitet som er den samlede mængde fra en maksimal udånding fra en maksimal indånding, og er normalt 4,8 liter. 
Vi kunne se at de folk der motionerede havde overalt bedre resultater end folk der ikke motionerede. Det er meget vigtigt at have gode resultater her da det kan være livsfarligt at klare sig værre. At have en for høj puls er ikke godt da hjertet skal arbejde mere, siden det skal pumpe hårdere for få blodet rundt i kroppen. Det er også vigtigt at have et blodtryk tæt på gennemsnittet da hjertet igen skal arbejde hårdere for at få blodet sendt rundt i kroppen. Det er også godt at have et lavt fedtprocent da det betyder at man er sundere. Det betyder også at ens krop skal arbejde mere, da den vejer mere. BMI er en af de mindre troværdige faktorer da et højt BMI ikke er ensbetydene med at man er usund. For eksempel kan man have kæmpe muskler og derfor vejer mere end man burde. For eksempel vejer Arnold 107 kg og er 188cm høj. Det betyder at han har en BMI på 30,27 dog kan man ikke sige at han er usund. Det er også vigtigt at have en stor vitalkapacitet da det lader en optage mere ilt og bruge mere ilt til at præstere bedre.  

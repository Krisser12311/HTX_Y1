\newpage
\part{Kulstofkredsløb}
\subsection*{Redegør for hvordan carbon/energi flyttes rundt i et kredsløb}
\subsection*{Forklar processerne fotosyntese og respiration, herunder indragelse af øvelsen: Fotosyntese og respiration}
\subsection*{Diskuter hvordan fotosyntese og respiration spiller en vigtig rolle for økosystemer og klimaet}
Energi og carbon flyttes rundt i et kredsløb ved brug af forskellige processer. I planter sker det under fotosyntesen og respiration. Her bliver kuldioxid, vand og lysenergi lavet om til ilt og glucose, og under respirationen bliver glucose og ilt lavet om til carbondioxid, vand og energi i form af ATP. Måden carbon flytter sig rundt i planter er ved at planter starter med at optage kuldioxid og ilt under fotosyntesen. Som sat sammen vil blive til glucose, dog før kuldioxid kan bruges til at danne glucose skal oxygen atomerne spaltes fra. Cirka halvdelen af den dannede glucose bliver brugt som brændstof til respiration for at danne energi i form af ATP. Her bliver kuldioxid dannet igen sammen med vand og energi i form af ATP. Kort sagt er kulstofkredsløbet i planter, uorganisk stof (\begin{math}CO_2\end{math}) til organisk stof (Glucose) tilbage til uorganisk stof (\begin{math}CO_2\end{math}).

Disse planter bliver nok spist af planteædere, som er derfor heterotrofer, da de ikke selv kan lave deres egen mad. Planterne bliver spaltet til små molekyler som bliver optages i organismens blod og organismens celler bruger molekylerne som både brændstof og byggesten. Ligesom i planter, bliver glucosen som er blevet optaget brugt under respiration for at danne energi i form af ATP under respiration. 
Det næste trin i kulstofkredsløbet er at når autotrofer og/eller heterotrofe dyr dør bliver de spist af nedbrydere. Eksempler på nedbrydere er svampe, regnorme osv.  De lever af for det meste dødt organisk stof, som bliver brugt på præcist samme måde som i autotrofer og heterotrofer. Til sidst bliver kuldioxiden frigjort til atmosfæren igen. Derfor kan man sige at kulstofkredsløbet går som følgene: Atmosfæren→Autotrofer→Heterotrofer→Nedbrydere→Atmosfæren

Respiration og fotosyntese er to biokemiske processor som planter går igennem dagligt. Dog er det ikke kun planter der går gennem respiration, næsten alle levende organismer respirer, og det er heller ikke kun planter der fotosyntetiserer, der er et par bakterier som også gør som fx cyanobakterier.
Inde i eukaryote planteceller er der et specielt organel som kaldes grønkorn. Grønkornet indeholder chlorofyl og carotenoider. Inde i grønkornet foregår 15-20 mindre processor, som kan opdeles i to processer, lysprocesser og mørkeprocesser, processerne i helhed kaldes fotosyntese. Fotosyntesen kan beskrives som \begin{equation} 6CO_2 + 6H_2O \rightarrow C_6H_{12}O_6 + 6O_2 \end{equation}6 hvilket siger at vand, kuldioxid og lysenergi bliver lavet om til ilt og glucose. Her bliver lysenergi transformeret til kemisk energi i form af glucose. 
Lysprocceserne foregår i et membranstruktur som kaldes thylakoid, og processerne kan kun foregå mens planten absorberer lys. 
Processerne har brug for chlorofyl da det kan absorbere lysenergi og en masse enzymer som bliver brugt som katalysator i nogle af lysprocesserne. Kort sagt kan lys processerne beskrives som 
\begin{math}
    2H_2 O+lys \rightarrow O_2+4H
\end{math}. Under processen bliver vand spaltet og ilt bliver dannet. 
Hydrogenet fra vandet bliver overført til \begin{math}NADP^+\end{math} hvilket bliver til NADPH. Lysenergien bliver brugt til at sammensætte ADP og \begin{math}P_i\end{math} til ATP (adenosintrifosfat) hvilket betuder at lysenergi bliver transformeret til kemisk energi i form af ATP. 
Mørkeprocesserne foregår ikke kun mens planten ikke absorberer lys, dog kan de godt foregår i mørket. Under processerne optager planten kuldioxid, som kombineres med hydrogen fra NADPH og glucose bliver dannet. Mørkeprocesserne kan som helhed beskrives som \begin{math}4H+CO_2 \rightarrow CH_2 O+H_2 O\end{math}.
Når man lægger de to reaktioner sammen får man \begin{math}
    6CO_2+6H_2 O+lys \rightarrow C_6 H_{12} O_6+6O_2
\end{math}  og når det bliver afstemt får man  \begin{math}6CO_2+6H_2 O+lys \rightarrow C_6 H_12 o_6+6O_2\end{math}.

Inde i eukaryote celler er der et organel som kaldes et mitokondrie, og respiration foregår derinde. ATP er den eneste form for energi som celler kan bruge til de diverse processer, men en celle har en meget lille mængde af ATP, derfor er det meget vigtigt at ATP bliver genopbygget ofte. Genopbygningen af ATP i en celle foregår i en masse biokemiske processer som i helhed kaldes respiration. 
Når ATP bliver spaltet frigives energi som bliver brugt af cellen og derfor kræver det også energi at genopbygge ATP. Energien til at genbygge ATP bliver lavet når glucose reagerer med ilt, hvor der bliver dannet kuldioxid og vand. Derfor kan respiration beskrives som \begin{equation}C_6H_{12}O_6 + 6CO_2 \rightarrow 6O_2 + 6H_2O + ATP \end{equation}.
I øvelsen Fotosyntese og Respiration hos vandpest skulle vi bevise at planter fontosyntetiserer og respirerer. Vi gjorde det ved at finde  8 reagensglas, hvor vi hældte carboneret vand i fire af dem og normalt vand i de sidste fire. Vi puttede vandpest i fire af reagensglas og efterlod fire uden vandpest og vi lod også kun fire af reagensglasene stå i sollyset og de andre fire lod vi stå i mørket uden noget lys ved at dække dem med staniol. Efter vi havde puttet alle forskellige materiale ind i de diverse reagensglas puttede vi tape ved åbningen af reagensglassne for at sikre os at der ikke ville være nogle eksterne faktorerer som ville ændre resultaterne af vores forsøg og også for at sikre os at kuldioxiden og ilten ville forblive inde i reagensglasene. 
For at finde ud om planter respirerer og fotosyntetiserer skulle vi kigge på pH-værdien af vandet i de forskellige glas efter noget tid var gået. 
Vi brugte pH-indikatoren bromthymolblåt, hvor vandet ville være blåt hvis væsken var basisk og gul hvis væsken var sur. Man kan også sige at hvis mængden af \begin{math}CO_2\end{math} i vandet stiger så ville vandet bliver mere surt, og derfor kan man også sige at hvis mængden af \begin{math}CO_2\end{math} falder i vandet så ville vandet være mere basisk. Det betyder at hvis man havde et reagensglas med vandpest, \begin{math}CO_2\end{math} og lys, så burde farven af vandet ændre sig fra gul til blå. Selvom min gruppe ikke fik den til at gå fra gul til blå, gik den fra en meget intens gul, til en meget mindre intens gul. 

De to processer, fotosyntese og respiration, har en kæmpe indflydelse på klimaet og økosystemer. Fotosyntese har især en stor indflydelse da det hjælper med at formindske drivhuseffekten ved at absorbere noget kuldioxid og lave det om til ilt som vi kan bruge. Respirationen har også en stor indflydelse på klimaet da man helst vil have en balance på kuldioxid niveaut. Hvis der ikke var noget kuldioxid i atmosfæren, så ville Jorden være ubeboelig for mange levende væsner, da temperaturen ville falde ekstremt. Dog på samme tid er det også vigtigt at der ikke er for meget kuldioxid i atmosfæren, da det ville betyde at den globale gennemsnitstemperatur ville stige og på et tidspunkt blive alt for høj til at understøtte de fleste levende organismer. Respiration og fotosyntese hjælper også klimaet da det lader planten leve, det er en meget god ting for alt levende på Jorden, da ikke alle levende arter på Jorden kan fotosyntetisere for at lave glucose. Glucose er nødvendigt for alt levende da det lader organismen genopbygge ATP og derfor leve i længere tid.
Kort sagt kan man sige at fotosyntesen hjælper miljøet med at formindske niveauet af kuldioxid i atmosfæren og forbedrer økosystemer ved at producere ilt som andre levende organismer bruger for at overleve, og respiration hjælper med at få niveauet af kuldioxid i atmosfæren til at stige og forbedrer økosystemer ved at lade planter overleve, hvilket hjælper da planter er for det meste autotrofe og plejer typisk at være i bunden af fødekæden.

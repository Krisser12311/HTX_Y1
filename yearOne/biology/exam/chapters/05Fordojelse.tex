\newpage
\part{Fordøjelse}
\section*{Redegør for fordøjelsessystemets opbygning og funktion.}
\section*{Forklar øvelsen: Bromelin i Ananas og relater til fordøjelsessystemet.}
\section*{Diskuter hvordan blodsukker reguleres, og hvilke konsekvenser Diabetes kan have for mennesker.}
Fordøjelsessystemet er en af de mange forskellige organsystemer mennesker har, med formål om at indtage, transportere, nedbryde og optage livsvigtige stoffer. 
Selve fordøjelsesprocessen starter i mundhulen. Her bliver føden forabejdret af tænderne som foretager en mekanisk bearbejdning af føden. Kindtænderne bliver brugt til at male og knuse, mens fortænderne bliver brugt til at klippe og skære. Spytkirtlerne udskiller et fordøjelsesenzym i form af spytamlyse, det starter den kemiske fordøjelse og nedbryder amylose eller stivelse til kortkædede kulhydrater. Tungen bliver brugt til at hjælpe med at mekanisk bearbejdre føden og hjælper også med synkeprocessen. I svælget bliver føde og luft adskillet og føden bliver ved brug af peristaltiske bevælgelser sendt videre til mavesækken.
I mavesækken bliver proteiner fra føden brudt ned til kortkædede peptider. Processen bliver katalyseret af enzymet pepsin. Pepsin virker bedst i et miljå med en pH-værdi på 2-3, det er meget heldigt da mavesækken har et surt miljø grundet udskillese af saltsyre. Dog bliver saltsyren ikke kun brugt til at stimulere dannelsen af pepsin, men også som et bakteriedræbende middel. Når der er proteiner, peptider eller alkohol i mavesækken laver nogle lukkede kirtler hormonet gastrin, som fremmer produktionen af saltsyre.
Tyndtarmen er et meget foldet og 5 meter lang organ som er en del af fordøjelsessystemet. De største foldninger kaldes villi og er lavet af celler, hvis membran også er foldet i hvad kaldes mikrovilli. Det betyder at tyndtarmens overfladeareal bliver meget højere og derfor bliver evnen til at optage forskellige stoffer bedre. Den første del af tyndtarmen kaldes for tolvfingertarmen, da den er tolv fingre lang. I tolvfingertarmen findes gange som bliver brugt til at udskille væske fra bugspytkirtelen og galdeblærne. Bygspyttet indeholder mange forskellige enzymer, der bliver brugt til at katalyserer spaltningen af kulhydrater, proteiner og fedt. Galdeblæren indeholder galde som er dannet i leveren. Dens funktion er at findele større fedtdråber til mindre dråber så lipasen har lettere adgang til at spalte lipiderne. Kort sagt kan man sige at i tyndtarmen bliver kulhydrater spaltet til monosaccharider, proteiner bliver spaltet til aminosyrer, nucleinsyrer bliver spaltet til nucleotider og fedt bliver spaltet til glycerol og fedtsyrer. Alle af de førnævnte stoffer bliver derefter optaget til blodet eller lymfen ved brug af enten aktiv eller faciliteret transport.
I tyktarmen bliver vand og salte optages. Der findes også en masse forskellige bakterier så som colibakterer. De bliver brugt til at forgære en del af de ufordøjelige stoffer (cellulose) i føden og danner noget k-vitamin. K-vitamin bliver brugt til at forbedre blodets evne til at koagulere. Dagligt bliver der dannet 0,5-3 liter luft under forgæringen, som bliver lukket ud i form af prutter af endetarmen. De ufordøjelige madrester, tarmbakterier og nogle døde celler fra tarmens slimhinde kommer ud fra systemet fra endetarmen i form af afføring.

I øvelsen bromelin i ananas skulle vi klargøre en masse reagensglas med vand og opløst gelatine i alle. I en af reagensglassene skulle vi tilsætte noget frisk ananasjuice fra en frisk presset ananas. Vi skulle også tilsætte konserveret ananassaft, ananasjuice og ananassaft fra dåse. I min gruppe puttede vi også saltsyre sammen med ananassaft for at se hvordan det ville påvirke gelatinens evne til at størkne. Formålet med øvelsen var at finde ud af hvornår et enzym, bromelin, ville hindre gelatine, som består af protein, i at størkne og danne gele. Vi gjorde det for at se om bromelin ville blive denatureret i forskellige miljø og hindre det i at virke.
Dette forsøg relater til fordøjelsessystemet ved at vise hvordan enzymer i kroppen bliver brugt til at nedbryde proteiner. Vi kunne også se at det var meget vigtigt at miljøet passer til enzymet da det ellers ville denaturer enzymet og enzymet ikke ville kunne fungerer. Vi kunne se at enzymer bliver brugt til at nedbryde nogle proteiner ved at observere at i nogle reagensglas kunne geleet ikke størkne grundet bromelin.

Blodsukker bliver reguleret i kroppen ved at kulhydrater bliver nedbrydet til først oligosaccharider og derefter til disaccharider. Disse disaccharider bliver spaltet til glucose i tyndtarmen. Glucosen bliver optages i blodkarrene, som ligger op af tyndtarmscellerne. Det betyder at koncentrationen af glucosen stiger hurtigt efter indtagelsen af kulhydrater. Det bliver registeret i bugspytkirtlernens insulinproducerende celler som udgiver et hormon, insulin. Insulin er et hormon der er opbugget af 51 aminosyrer, det bliver brugt til at fremme cellernes evne til at optage og lagre kulydrat, fedtstof og protein. Når glucose bliver transporteret rundt i cellerne falder blodets koncentration af glucose, blodsukker, og derfor er insulin et blodsukkerregulerende hormon.
Diabetes er en sygdom som er karakteriseret ved den manglende evne til at regulere koncentrationen af glucose i blodet. I et rask mennesker er koncentrationen af glucose på de cirka 48 mmol per liter blod. Diabetes har derfor store konsekvenser for mennesker da de ikke kan regulere kroppens koncentration af glucose i blodet. Det er dårligt da insulin bliver også brugt til at føre glucose ind i cellerne så cellerne kan bruge det som brændstof. Det er ikke godt for mennesker da vi har brug for energi for at leve og glucose er en af de vigtige energikilder vi har.
(SE EVT Tegninger i notesbog)
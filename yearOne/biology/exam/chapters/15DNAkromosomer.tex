\newpage
\part{DNA og kromosomer}
\subsection*{Redegør for hvordan kromosomer og DNA er opbygget}
\subsection*{Forklar de 2 celledelinger mitose og meiose og diskuter hvordan mutationer opstår}
\subsection*{Redegør for øvelsen DNA fra løg og kom ind på mulig anvendelse af udvundet DNA}
DNA (de-oxy-ribo-nuclein-syre) er opbygget af nucleotider, som er sat sammen i lange kæder som kaldes polynucleotid-kæder. 
Nucleotid består af tre forskellige dele: en phosphat gruppe, et kulhydrat og en nitrogenholdig base (Adenin, Guanin, Cytosin og Thymin)
DNA er opbygget af to antiparallele strenge som snor sammen til at danne en dobbelthelix. Strengene bliver holdt sammen af hydrogenbindinger mellem de forskellige nitrogenholdige baser. Adenin med Thymin og Guanin med Cytosin.
Dannelsen af en af strengene starter altid ved en phosphat gruppe på 5 mærke positionen i kulhydrat-ringen i det nukleotid hvor strengen bliver dannet, og slutter ved at blive bundet til stregen som vokser ved brug af OH (Hydroxid) gruppen som sidder på 3 mærke.
Kromosomer er lavet af DNA. Dog er det DNA som er snoet rundt om proteiner, som kaldes histoner. Når DNA’et er snoet rundt om 8 histoner bliver der lavet en nukleosom, og når disse nukleosomer er pakket tæt bliver de kaldt kromosomer.

Mitose og meiose er de to forskellige måder en celle kan dele sig. Dog er det ikke tilfældigt hvilken af de to måder cellen kommer til at dele sig på. Hvis cellen er en kønscelle vil den gå igennem meiose for at dele sig og hvis cellen ikke er en kønscelle vil den gå gennem mitose for at dele sig.
I mitosen er der fem forskellige faser som en celle går igennem før den bliver delt og har lavet flere celler. Interfasen, profasen, metafasen, anafasen og telofasen.
\begin{longtable}{ | m{2cm} | m{15cm}|}
    \caption{Mitose} \\
    \hline
    Fase & Beskrivelse \\ \hline
    Interfasen & Interfasen er den fase som cellen bruger det meste af tiden i, da det er den fase hvor cellen bare laver de normalle ting en celle gør. Dog når cellen kommer tættere på at dele sig selv begynder DNA-strengene (kromatin) at strække sig ud, centrosomer og DNA begynder også at kopierer sig selv. \\ \hline 

    Profasen & I profasen begynder kromatinet at kondensere og begynde at ligne kromosomer. Disse kromosomer består af to identiske søsterkromatider, som er forbundet i midten ved et centromer. Cellekernen, som indeholder alt af en celles DNA, begynder også at blive nedbrudt i denne fase og centrosomerne flytter sig hen til forskellige sider af cellen \\ \hline 

    Metafasen & I metafasen, nu hvor kromosomerne er fuldt kondenseret, begynder kromosomerne at flytte sig ind til midten af cellen, for at danne en lige linje. De danner en lige linje for at sikre sig at søsterkromatiderne kan blive skilt ordenligt i den næste fase af mitosen. \\ \hline 

    Anafasen &  Under anafasen bliver kromosomerne skilt og de to søsterkromatider som kromosomerne bestod af bliver trukket til hver ende af cellen. Selve cellen begynder også at strække sig og cellen bliver tættere på at dele sig helt. \\ \hline 

    Telofasen &  Under telofasen bliver selve cellen splittet og bliver lavet om til to identiske datterceller. Kromosomerne begynder også at returnere til deres tidligere form som kromatin og forskellige organeller i cellen, ligesom cellekernen, begynder at blive genformet. \\ \hline 
\end{longtable}

Meiosen er en måde at en celle kan dele sig, dog er det kun kønsceller eller gameter, som går igennem meiosen. Der er ni faser i meiosen, som er interfasen, profasen I, metafasen I, anafasen I, telofasen I, profasen II, metafasen II, anafasen II og telofasen II.
\begin{longtable}{ | m{3cm} | m{14cm} |}
    \caption{Meiose} \\
    \hline
    Fase & Beskrivelse \\ \hline
    
    Interfase & Interfasen under meiose er præcist den samme som under mitosen. Cellen begynder at kopierer dens DNA og  gør sig klar til at begynde processen. \\ \hline

    Profase & Den første gang cellen går igennem profasen ligner fasen meget profasen i mitose, da kromatinet begynder at kondensere og begynder også at ligne kromosomer. Dog bliver kromosomernes genetiske materiale udvekles med andre kromosomer. Det gør så cellen opår genetisk variation. \\ \hline

    Metafase &  I den første metafase blier kromosomerne arrageret i par og bliver fastgjort til spindelapparatet for at gøre dem klar til at blive trukket til begge sider af cellen. \\ \hline

    Anafase & I anafase I bliver kromosomparrene delt og trukket til begge sider. Dog ligesom i mitosen bliver kromosomerne skilt og de to kromatider som kromosomerne bestod af bliver også trukket til hver side af cellen. \\ \hline

    Telofase & I telofase I bliver cellerne delt helt, hvilket betyder at der nu er to datterceller som er haploide, hvilket betyder at deres kromosomantal er halveret. Nye cellekerne bliver også dannet og kromosomerne begynder at dekondensere. \\ \hline

   Profase II & Igen i profase II begynder kromosomerne at blive kondenseret og cellekernet bliver også nedbrudt igen. Centriolerne begynder også at flytte sig hen til modsatte sider af cellen for at gøre klar til de næste faser. \\ \hline
 
    Metafase II &  Kromosomerne begynder at bevæge sig ind mod midten af cellen i en lige linje og kromosomerne gør sig klar til at blive adskilt i den næste fase. \\ \hline

    Anafase II & Kromosomerne begynder at bevæge sig ind mod midten af cellen i en lige linje og kromosomerne gør sig klar til at blive adskilt i den næste fase. \\ \hline

    Telofase II & I telofase II bliver cellerne delt igen hvilket betyder at der nu er fire celler til sidst. Nye cellekerne bliver også genbygget inde i cellerne for at holde på en celles DNA. Kromosomerne dekondenserer også igen og meiosen er færdig efter dette trin. \\ \hline

\end{longtable}

\begin{longtable}{ | m{5cm} | m{5cm} | m{5cm} | }
    \hline 
    & Meiose & Mitose \\ \hline
    Genetisk variation & Ja & Nej \\ \hline
    Antal datterceller & 4 & 2 \\ \hline
    Type celle & kønsceller & Somatiske celler \\ \hline
    mængde divisioner & 2 & 1 \\ \hline
\end{longtable}
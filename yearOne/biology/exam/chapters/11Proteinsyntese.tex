\newpage
\part{Proteinsyntese}
\subsection*{Forklar hvad DNA, kromosomer og gener er, samt hvordan de er opbygget.}
\subsection*{Redegør for hvordan der dannes protein ud fra gener.}
\subsection*{Med udgangspunkt i øvelsen: DNA fra løg, ønskes en diskussion af anvendelsen af DNA.}
DNA (de-oxy-ribo-nuclein-syre) er opbygget af nucleotider, som er sat sammen i lange kæder som kaldes polynucleotid-kæder.  Nucleotider består af tre forskellige dele: en phosphat gruppe, et kulhydrat og en nitrogenholdig base (Adenin, Guanin, Cytosin og Thymin). DNA er opbygget af to antiparallele strenge som snor sammen til at danne en dobbelthelix. Strengene bliver holdt sammen af hydrogenbindinger mellem de forskellige nitrogenholdige baser. Adenin med Thymin og Guanin med Cytosin. Dannelsen af en af strengene starter altid ved en phosphat gruppe på 5 mærke positionen i kulhydrat-ringen i det nukleotid hvor strengen bliver dannet, og slutter ved at blive bundet til stregen som vokser ved brug af OH (Hydroxid) gruppen som sidder på 3 mærke.
Kromosomer er struktuerer i cellerne lavet af DNA og proteiner. Dog er det DNA som er snoet rundt om proteiner, som kaldes histoner. Når DNA’et er snoet rundt om 8 histoner bliver der lavet en nukleosom, og når disse nukleosomer er pakket tæt bliver de kaldt kromosomer. Et menneske har normalt 46 kromosomer. 
Gener er opbygget af en af specifik sekvens af baser, som indeholder information, der fortæller cellen at den skal producere et protein baseret på sekvensen af baser. For at kunne lave proteinet skal cellen gøre flere ting, den skal kopierer DNA sekvensen til mRNA og derefter oversætte det til et protein ved brug af cellens ribosomer. De er ansvarlige for arvelighed og overførsel af træk fra ens forældre til en selv. Gener er unikke for hver organisme og bestemmer træk. 
Disse gener bliver brugt til at lave forskellige proteiner som bliver brugt rundt i kroppen. Processen for produktionen af DNA til protein bliver typisk kaldt Det centrale dogme. Processen siger at DNA bliver til RNA som bliver til protein. Første trin kaldes transkription og det andet trin kaldes translation.
Under transkription bliver længden af DNA sekvensen brugt til at bestemme hvor lang selve protein kæden kommer til at blive når hele processen er færdig. I denne fase bliver DNA sekvensen omskrevet til hvad det ville svare til i en RNA sekvensm ved hjælp af RNA-polymerase. RNA er meget ligesom DNA, da det også er lavet af en kæde nucleotider, som også er lavet af en phosphat-gruppe, et kulhydrat og fire forskellige baser. Dog har RNA ikke thymine ligesom normalt DNA, men i stedet uracil. Det betyder at i RNA vil alle steder hvor der havde vøret thymine i DNA bliver lavet om til uracil i RNAet. 
Under translation bliver informationen på et mRNA-molekyle oversat til en rækkefølge af aminosyrer i et polypeptid. 
Dog svarer en base ikke til en aminosyre, da der kun findes 20 forskellige aminosyrer og 4 forskellige baser. Man har brug for tre baser til at danne en animosyre og siden der er fire forskellige baser kan man sige \begin{math}4^3=64\end{math}. Det betyder at der er 64 forskellige kombinationer af baserne, men siden der kun er 20 forskellige aminosyrer, kan nogle af dem have flere forskellige codons. Ved eukaryoter starter polypeptid kæden nærmest altid med codonnet AUG eller methionin og ved prokaryoter er det typisk GUG eller valin. 
Polypeptid kæden ved at den er færdig når den kommer mod en af de tre stop-codons, UAA, UAG eller UGA.
Aflæsningen af mRNA og oversættelsen af codons til aminosyrer forgår på ribosomerne, som er organeller der flyder rundt i cellens cytoplasma. De kaldes også celles proteinfabrik. Genkendelsen af dem forgår på et specialiseret RNA-molekyle som kaldes tRNA (transfer RNA) og for hvert animosyre er der en specifik tRNA-molekyle. TRNA binder de specielle aminosyrer i en ende og indeholder et anitcodon i den anden ende. Et anticodon baseparrer med codonnet i mRNA. For eksempel vil et anticodon til GAG være CUC.
Til selve translation er der tre forskellige faser gennem processen. Initering, elongering og terminering
\begin{longtable}{| m{2cm} | m{7cm} | }
    \hline 
    Fase & Beskrivelse \\ \hline
    Initering & mRNA bindes til ribosomet og start-codonnet bliver fundet. Methionin tilføres til et anticodon typisk UAC ved hjælp af tRNA. \\ \hline
    Elongering & I denne fase bliver de rigtige aminosyrer tilført via tRNA. Anticodonnet genkender de forskellige codonner og binder til de rigtige i mRNA'et. Mellem de tilførte aminosyrer og tRNA molekylerne dannes peptidbindinger, og der bliver gjort klar til at transportere en aminosyre til en anden proteinfabrik- \\ \hline
    Terminering & I denne fase bliver stop-codonnet aflæst og den dannede polypeptidkæde bliver frigjort fra ribosomerne, og når denne polypeptidkæde bliver foldet rigtigt bliver den lavet om til et protein. \\ \hline
\end{longtable}
I forsøget DNA fra løg skulle vi isolere DNA'et fra et løg ved at først nedbryde vævet ved brug af en blender. Vi skulle også fjerne cellevægge, cellemembraner og kernemembraner ved brug af et vaskemiddel. Efter det skulle vi filtrerer alt ubrugeligt materiale ved at bruge et kaffefilter. Eftersom at der også var proteiner skulle vi også fjerne dem da vi kun var interreseret i DNA'et, for at fjerne proteinerne skulle vi bruge et protease enzym. Til sidst hældte vi iskold ethanol ned af siden af reagensglasset for at danne et lag af ethanol over løg-ekstraktet. 
Der er mange mulige anvendelser af udvundet DNA, nogle af dem jeg kan tænke på er at sekvensere DNA. Det er rimelig brugbart da det lader os finde ud af hvilke genetiske variationer øger farlige sygdomer og andre ting som man helst vil undgå. En anden anvendelse af udvundet DNA er at man kan analyserer planters DNA før de bliver planet for at se hvilke planter vil være bedre at plante og dermed foretage selektiv avl.
